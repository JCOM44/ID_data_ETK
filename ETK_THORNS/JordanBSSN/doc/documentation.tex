% *======================================================================*
%  Cactus Thorn template for ThornGuide documentation
%  Author: Ian Kelley
%  Date: Sun Jun 02, 2002
%  $Header$
%
%  Thorn documentation in the latex file doc/documentation.tex
%  will be included in ThornGuides built with the Cactus make system.
%  The scripts employed by the make system automatically include
%  pages about variables, parameters and scheduling parsed from the
%  relevant thorn CCL files.
%
%  This template contains guidelines which help to assure that your
%  documentation will be correctly added to ThornGuides. More
%  information is available in the Cactus UsersGuide.
%
%  Guidelines:
%   - Do not change anything before the line
%       % START CACTUS THORNGUIDE",
%     except for filling in the title, author, date, etc. fields.
%        - Each of these fields should only be on ONE line.
%        - Author names should be separated with a \\ or a comma.
%   - You can define your own macros, but they must appear after
%     the START CACTUS THORNGUIDE line, and must not redefine standard
%     latex commands.
%   - To avoid name clashes with other thorns, 'labels', 'citations',
%     'references', and 'image' names should conform to the following
%     convention:
%       ARRANGEMENT_THORN_LABEL
%     For example, an image wave.eps in the arrangement CactusWave and
%     thorn WaveToyC should be renamed to CactusWave_WaveToyC_wave.eps
%   - Graphics should only be included using the graphicx package.
%     More specifically, with the "\includegraphics" command.  Do
%     not specify any graphic file extensions in your .tex file. This
%     will allow us to create a PDF version of the ThornGuide
%     via pdflatex.
%   - References should be included with the latex "\bibitem" command.
%   - Use \begin{abstract}...\end{abstract} instead of \abstract{...}
%   - Do not use \appendix, instead include any appendices you need as
%     standard sections.
%   - For the benefit of our Perl scripts, and for future extensions,
%     please use simple latex.
%
% *======================================================================*
%
% Example of including a graphic image:
%    \begin{figure}[ht]
% 	\begin{center}
%    	   \includegraphics[width=6cm]{MyArrangement_MyThorn_MyFigure}
% 	\end{center}
% 	\caption{Illustration of this and that}
% 	\label{MyArrangement_MyThorn_MyLabel}
%    \end{figure}
%
% Example of using a label:
%   \label{MyArrangement_MyThorn_MyLabel}
%
% Example of a citation:
%    \cite{MyArrangement_MyThorn_Author99}
%
% Example of including a reference
%   \bibitem{MyArrangement_MyThorn_Author99}
%   {J. Author, {\em The Title of the Book, Journal, or periodical}, 1 (1999),
%   1--16. {\tt http://www.nowhere.com/}}
%
% *======================================================================*

% If you are using CVS use this line to give version information
% $Header$

\documentclass{article}

% Use the Cactus ThornGuide style file
% (Automatically used from Cactus distribution, if you have a
%  thorn without the Cactus Flesh download this from the Cactus
%  homepage at www.cactuscode.org)
\usepackage{../../../../doc/latex/cactus}

\begin{document}

% The author of the documentation
\author{Miguel~Zilhão and Helvi~Witek}

% The title of the document (not necessarily the name of the Thorn)
\title{\texttt{LeanBSSNMoL}: An Einstein Toolkit thorn for evolving Einstein's equations in the BSSN formalism with the \texttt{MoL} thorn}

% the date your document was last changed, if your document is in CVS,
% please use:
%    \date{$ $Date$ $}
% when using git instead record the commit ID:
%    \date{\gitrevision{<path-to-your-.git-directory>}}
\date{June 30, 2023}

\maketitle

% Do not delete next line
% START CACTUS THORNGUIDE

% Add all definitions used in this documentation here
%   \def\mydef etc

% Add an abstract for this thorn's documentation
\begin{abstract}
\texttt{LeanBSSNMoL} solves Einstein's equations in the BSSN formalism with up
to 6th order accurate finite-difference stencils. It relies on the
\texttt{MoL} thorn for the time integration.
\end{abstract}

\section{\texttt{LeanBSSNMoL}}

The \texttt{Lean} code was first introduced in~\cite{Sperhake:2006cy}, to evolve vacuum spacetimes with the BSSN formalism.
%
It has since been modified to run with the \texttt{MoL} thorn for the time integration and generalized to include matter terms through the \texttt{TmunuBase}
thorn. It was made publicly available through the \texttt{Canuda} numerical relativity library~\cite{Canuda}, and has since been distributed also as a part of the Einstein Toolkit.

The bulk of the code is written in Fortran~90 and should be simple to follow -- emphasis has been given to readability. The main part of the code, where the right-hand side of the evolution equations is computed, can be found in the file \texttt{calc\_bssn\_rhs.F90}.
%
For the conformal factor the code uses the ``W'' version, i.e.\ $W=\gamma^{-1/6}$, and it employs the usual ``1+log'' and ``Gamma-driver'' gauge conditions.


\section{Obtaining this thorn}

\texttt{LeanBSSNMoL} is included with the Einstein Toolkit and can also be obtained through the \texttt{Canuda} numerical relativity library~\cite{Canuda}.


\begin{thebibliography}{9}

%\cite{Sperhake:2006cy}
\bibitem{Sperhake:2006cy}
U.~Sperhake,
``Binary black-hole evolutions of excision and puncture data,''
Phys. Rev. D \textbf{76} (2007), 104015
doi:10.1103/PhysRevD.76.104015
[arXiv:gr-qc/0606079 [gr-qc]].

\bibitem{Canuda}
H.~Witek, M.~Zilhao, G.~Bozzola, C.-H.~Cheng, A.~Dima, M.~Elley, G.~Ficarra, T.~Ikeda, R.~Luna, C.~Richards, N.~Sanchis-Gual, H.~Okada~da~Silva.
``Canuda: a public numerical relativity library to probe fundamental physics,''
Zenodo (2023)
doi: 10.5281/zenodo.3565474

\end{thebibliography}

% Do not delete next line
% END CACTUS THORNGUIDE

\end{document}
