\documentclass{article}
%\usepackage{../../../../doc/ThornGuide/cactus}
\usepackage{../../../../doc/latex/cactus}
\begin{document}

% The title of the document (not necessarily the name of the Thorn)
\title{The {\tt GRHydro} code: three-dimensional relativistic hydrodynamics}

% The author of the documentation - on one line, otherwise it does not work
\author{Original authors: Luca Baiotti, Ian Hawke, Pedro Montero \cr Contributors: 
  Sebastiano Bernuzzi, Giovanni Corvino, Toni Font, Joachim Frieben, \cr Roberto De Pietri, Thorsten
  Kellermann, Frank L\"offler, Christian D. Ott, \cr Luciano Rezzolla, Nikolaos Stergioulas, Aaryn
  Tonita, \cr {\it and many others,} \cr {\it especially those who were in the European Network ``Sources of
    Gravitational Waves'' } }

% the date your document was last changed, if your document is in CVS, 
% please use:
\date{Date: 2009-12-07 19:20:47 -0600 (Mon, 07 Dec 2009)}
\maketitle

% *======================================================================*
%  Cactus Thorn template for ThornGuide documentation
%  Author: Ian Kelley
%  Date: Sun Jun 02, 2002
%  $Header$                                                             
%
%  Thorn documentation in the latex file doc/documentation.tex 
%  will be included in ThornGuides built with the Cactus make system.
%  The scripts employed by the make system automatically include 
%  pages about variables, parameters and scheduling parsed from the 
%  relevent thorn CCL files.
%  
%  This template contains guidelines which help to assure that your     
%  documentation will be correctly added to ThornGuides. More 
%  information is available in the Cactus UsersGuide.
%                                                    
%  Guidelines:
%   - Do not change anything before the line
%       % START CACTUS THORNGUIDE",
%     except for filling in the title, author, date etc. fields.
%        - Each of these fields should only be on ONE line.
%        - Author names should be sparated with a \\ or a comma
%   - You can define your own macros are OK, but they must appear after
%     the START CACTUS THORNGUIDE line, and do not redefine standard 
%     latex commands.
%   - To avoid name clashes with other thorns, 'labels', 'citations', 
%     'references', and 'image' names should conform to the following 
%     convention:          
%       ARRANGEMENT_THORN_LABEL
%     For example, an image wave.eps in the arrangement CactusWave and 
%     thorn WaveToyC should be renamed to CactusWave_WaveToyC_wave.eps
%   - Graphics should only be included using the graphix package. 
%     More specifically, with the "includegraphics" command. Do
%     not specify any graphic file extensions in your .tex file. This 
%     will allow us (later) to create a PDF version of the ThornGuide
%     via pdflatex. |
%   - References should be included with the latex "bibitem" command. 
%   - use \begin{abstract}...\end{abstract} instead of \abstract{...}
%   - For the benefit of our Perl scripts, and for future extensions, 
%     please use simple latex.     
%
% *======================================================================* 
% 
% Example of including a graphic image:
%    \begin{figure}[ht]
%       \begin{center}
%          \includegraphics[width=6cm]{MyArrangement_MyThorn_MyFigure}
%       \end{center}
%       \caption{Illustration of this and that}
%       \label{MyArrangement_MyThorn_MyLabel}
%    \end{figure}
%
% Example of using a label:
%   \label{MyArrangement_MyThorn_MyLabel}
%
% Example of a citation:
%    \cite{MyArrangement_MyThorn_Author99}
%
% Example of including a reference
%   \bibitem{MyArrangement_MyThorn_Author99}
%   {J. Author, {\em The Title of the Book, Journal, or periodical}, 1 (1999), 
%   1--16. {\tt http://www.nowhere.com/}}
%
% *======================================================================* 

% If you are using CVS use this line to give version information
% $Header$

% Use the Cactus ThornGuide style file
% (Automatically used from Cactus distribution, if you have a 
%  thorn without the Cactus Flesh download this from the Cactus
%  homepage at www.cactuscode.org)

% Do not delete next line
% START CACTUS THORNGUIDE

% Add all definitions used in this documentation here 
%   \def\mydef etc

%\newcommand{\eqref}[1]{(\ref{#1})}

% Add an abstract for this thorn's documentation
\begin{abstract}
  {\tt GRHydro} is a fully general-relativistic three-dimensional hydrodynamics code. It evolves the
  hydrodynamics using High Resolution Shock Capturing methods and can
  work with realistic equations of state. The evolution of the
  spacetime can be done by any other ``appropriate'' thorn, such as
  the {\tt CCATIE} code, maintained and developed at the Albert-Einstein-Institut (Potsdam).
\end{abstract}

% The following sections are suggestive only.
% Remove them or add your own.

\section{Introduction}
\label{sec:intro}

The {\tt GRHydro}\footnote{GRHydro started as a branch of Whisky. 
  The name Whisky is
  due to Tom Goodale, who pointed out that the in the original Gaelic
  {\it uisge beatha} (from which {\tt Whisky} is derived) meant {\it water of
  life}. This name was chosen at a free and fair democratic ballot
  at the EU Network meeting in Southampton that gave the right answer
  thanks to a little bit of help from the authors of the code after
  some slight irregularities on the part of certain specialists in
  strange stars.} code is based upon the public version of the {\tt Whisky} code which
  is a product of the EU Network on Sources of
  Gravitational Radiation\footnote{http://www.eu-network.org},
  and was initially written by Luca Baiotti, Ian Hawke
  and Pedro Montero, based on the publicly available {\tt GR3D} code and
  with many other important contributors.
  With the help of large parts of the community, the {\tt GRHydro} code got
  improved, extended and included into the Einstein Toolkit.

\section{Using This Thorn}
\label{sec:use}

What follows is a brief introduction to using {\tt GRHydro}. It assumes that
you know the required physics and numerical methods, and also the
basics of Cactus\footnote{http://www.cactuscode.org}. If you don't, then skip this section and come back
to it after reading the rest of this ThornGuide of Cactus. For more details such
as thornlists and parameter files, take a look at the Einstein Toolkit web page
which is currently stored at
\begin{verbatim}
http://www.einsteintoolkit.org
\end{verbatim}

{\tt GRHydro} uses the hydro variables defined in {\tt HydroBase}
and provides own ``conserved'' hydro variables and methods to evolve them. It
does not provide any information about initial data or equations of
state. For these, other thorns are required. A minimal list of thorns
for performing a shock-tube test is given in the shock-tube test
parameter file, found at
\begin{verbatim}
GRHydro/test/GRHydro_test_shock.par
\end{verbatim}
and will include the essential thorns
\begin{verbatim}
GRHydro EOS_Omni ADMBase ADMCoupling MoL
\end{verbatim}
Initial data for shocks can be set using
\begin{verbatim}
GRHydro_Init_Data 
\end{verbatim}
Initial data for spherically symmetric static stars (with
perturbations or multiple ``glued'' stars) can be set by
\begin{verbatim}
TOVSolver
\end{verbatim}

The actual evolution in time is controlled by the Method of Lines
thorn MoL. For full details see the relevant ThornGuide. For the
purposes of {\tt GRHydro} at least two parameters are relevant; {\tt
  ode\_method} and {\tt mol\_timesteps}. If second-order accuracy is
all that is required then {\tt ode\_method} can be set to either {\tt
  "rk2"} (second-order TVD Runge-Kutta evolution) or {\tt "icn"}
(Iterative Crank Nicholson, number of iterations controlled by {\tt
  mol\_timesteps}, defaults to three), and {\tt mol\_timesteps} can be
ignored. A more generic (and hence less efficient) method can be
chosen by setting {\tt ode\_method} to {\tt "genrk"} which is a
Shu-Osher type TVD Runge-Kutta evolution. Then the parameter {\tt
  mol\_timesteps} controls the number of intermediate steps and hence
the order of accuracy. First to seventh order are currently supported.

{\tt GRHydro} currently uses a Reconstruction-Evolution type method. The type
of reconstruction is controlled by the parameter {\tt recon\_method}.
The currently supported values are {\tt "tvd"} for slope limited TVD
reconstruction, {\tt "ppm"} for the Colella-Woodward PPM method, and
{\tt "eno"} for the Essentially Non-Oscillatory method of Harten et
al. Each of these has further controlling parameters. For example
there are a number of slope limiters controlled by the keyword {\tt
  tvd\_limiter}, the PPM method supports shock detection by the
Boolean {\tt ppm\_detect}, and the ENO method can have various orders
of accuracy controlled by {\tt eno\_order}. Note that the higher-order
methods such as PPM and ENO require the stencil size to be increased
using {\tt GRHydro\_stencil} {\bf and} {\tt driver::ghost\_size}.

For the evolution various approximate Riemann solvers are available,
controlled by {\tt riemann\_solver}. Currently implemented are {\tt
  "HLLE"}, {\tt "Roe"} and {\tt "Marquina"}. For the Roe and Marquina
methods there are added options to choose which method is used for
calculating the left eigenvectors. This now defaults to the efficient
methods of the Valencia group, but the explicit matrix inversion is
still there for reference.

For the equations of state, two ``types'' are recognized, controlled
by the parameter {\tt GRHydro\_eos\_type}. These are {\tt "Polytype"}
where the pressure is a function of the rest-mass density, $P=P(\rho)$, and the
more generic {\tt "General"} type where the pressure is a function
of the rest-mass density and the specific internal energy, $P=P(\rho, \epsilon)$. For the
{\tt Polytype} equations of state one fewer equation is evolved and
the specific internal energy is set directly from the rest-mass density. The
actual equation of state used is controlled by the parameter {\tt
  GRHydro\_eos\_table}. Polytype equations of state include {\tt
  "2D\_Polytrope"} and general equations of state include {\tt
  "Ideal\_Fluid"}. 

\subsection{Obtaining This Thorn}

The public version of GRHydro can be found on the
website {\tt http://www.einsteintoolkit.org}. 

\subsection{Basic Usage}

The simplest way to start using {\tt GRHydro} would be to download some
example parameter files from the web page to try it. There are a number
of essential parameters which might reward some experimentation. These
include:
\begin{itemize}
\item Reconstruction methods:
  \begin{itemize}
  \item {\tt recon\_method}: The type of reconstruction method to
    use. {\tt tvd} is the standard. {\tt ppm} is more accurate but it requires 
    more resources. {\tt eno} gives in theory
    arbitrary order of accuracy but it is practically unworthy to go beyond fifth order.
  \item {\tt tvd\_limiter}: When using {\tt tvd} reconstruction the
    choice of limiter can have a large effect. {\tt vanleerMC2} is
    probably the best to use, but the extremes of {\tt minmod} and {\tt
      Superbee} are also interesting.
  \item {\tt ppm\_detect}: When using {\tt ppm} reconstruction with
    shocks, the shock detection algorithm can notably sharpen the
    profile.
  \end{itemize}
\item Riemann solvers: {\tt Marquina} is the standard solver
  used. {\tt HLLE} is significantly faster, but sometimes provides cruder approximation.
\item Equations of state: These are controlled by the {\tt
    GRHydro\_eos\_type} and {\tt GRHydro\_eos\_table} parameters. Changing
  these parameters will depend on which equation of state thorns you
  have compiled in.
\item Initial data parameters: {\tt GRHydro\_rho\_central} is inherited by many
  initial data thorns to set the central density of compact fluid
  objects such as single stars. However, this parameter is depreciated.
\item Atmosphere parameters: Many of these are listed in
  section~\ref{sec:atmosphere}. 
\end{itemize}

\subsection{Special Behaviour}

Although in theory {\tt GRHydro} can deal with conformal metrics as well as
physical metrics, this part of the code is completely untested as we
don't have conformal initial data (although this would not be hard -
we just haven't had the incentive).

\subsection{Interaction With Other Thorns}

{\tt GRHydro} provides the appropriate contribution to the stress energy
through the {\tt TmunuBase} interface. Those spacetime evolvers that
use this interface can use {\tt GRHydro} without change. 
%To pass the required variables across {\tt GRHydro} is a friend of {\tt ADMCoupling}.

{\tt GRHydro} uses the {\tt MoL} thorn to perform the actual time
evolution. This means that if all other evolution thorns are also
using {\tt MoL} then the complete evolution will have the accuracy of
the {\tt MoL} evolution method without change. This (currently) allows
for up to fourth-order accuracy in time without any changes to {\tt GRHydro}.

For the general equations of state {\tt GRHydro} uses the {\tt EOS\_Omni}
interface. This returns the necessary hydrodynamical quantities, such
as the pressure and derivatives with general function calls. The
parameter {\tt GRHydro\_eos\_table} controls which equation of state is
used during evolution.

For the metric quantities {\tt GRHydro} uses the standard {\tt
  CactusEinstein} arrangement, especially {\tt ADMBase}. This allows
the standard thorns to be used for the calculation of constraint
violations, emission of gravitational waves, location of the apparent
horizon, and more.

\subsection{Support and Feedback}

{\tt GRHydro} is part of the Einstein Toolkit, with its web page located at
\begin{verbatim}
http://www.einsteintoolkit.org
\end{verbatim}
It contains information on obtaining the code, together with
thornlists and sample parameter files. For questions, send an email
to the Einstein Toolkit mailing list {\tt users@einsteintoolkit.org}.

\section{Physical System}
\label{sec:phys}

The equations of general relativistic hydrodynamics can be written in
the flux conservative form
\begin{equation}
  \label{eq:consform1}
  \partial_t {\bf q} + \partial_{x^i} {\bf f}^{(i)} ({\bf q}) = {\bf
    s} ({\bf q}),
\end{equation}
where ${\bf q}$ is a set of {\it conserved variables}, ${\bf f}^{(i)}
({\bf q})$ the fluxes and ${\bf s} ({\bf q})$ the source
terms.
The 8 conserved variables are labeled $D$, $S_i$, $\tau$, and $\mathcal B^k$,
where
$D$ is the generalized particle number density, $S_i$ are the generalized
momenta in each direction, $\tau$ is an internal energy term, and $\mathcal
B^k$ is the densitized magnetic field.
These conserved variables are composed from a set of {\it primitive variables},
which are $\rho$, the rest-mass density, $p$, the
pressure, $v^i$, the fluid 3-velocities, $\epsilon$, the specific internal
energy, $B^k$ the magnetic field in the lab frame,
and $W$, the Lorentz factor, via the following relations
% from GRHydro/src/Prim2con.F90
%  w = 1.d0 / sqrt(1.d0 - (gxx*dvelx*dvelx + gyy*dvely*dvely + gzz &
%       &*dvelz*dvelz + 2*gxy*dvelx*dvely + 2*gxz*dvelx *dvelz + 2*gyz&
%       &*dvely*dvelz))
% 
%  dpress = (GRHydro_eos_gamma - 1.d0) * drho * deps
%
%  ddens = sqrt(det) * drho * w
%  dsx = sqrt(det) * (drho*(1+deps)+dpress)*w*w * (gxx*dvelx + gxy&
%       &*dvely + gxz*dvelz)
%  dsy = sqrt(det) * (drho*(1+deps)+dpress)*w*w * (gxy*dvelx + gyy&
%       &*dvely + gyz*dvelz)
%  dsz = sqrt(det) * (drho*(1+deps)+dpress)*w*w * (gxz*dvelx + gyz&
%       &*dvely + gzz*dvelz)
%  dtau = sqrt(det) * ((drho*(1+deps)+dpress)*w*w - dpress) - ddens
\
\begin{eqnarray}
  \label{eq:prim2con}
   D &=& \sqrt{\gamma}W\rho \nonumber \\
   S_i &=& \sqrt{\gamma} \left(\rho h^* W^{\,2} v_j-\alpha b^0b_j\right) \nonumber \\
   \tau &=& \sqrt{\gamma}\left(\rho h^* W^2 - p^*-(\alpha b^0)^2\right) - D, \nonumber \\
   \mathcal B^k &=& \sqrt{\gamma} B^k,
\end{eqnarray}
where $\gamma$ is the determinant of the spatial 3-metric $\gamma_{ij}$ and 
$h^* \equiv 1+\epsilon+\left(p + b^2\right)/\rho$, $p^* = p + b^2/2$.
$b^\mu$ is the magnetic field in the fluid's rest frame $b^\mu = u_\nu
^*F^{\mu\nu}$ where $^*F^{\mu\nu}$ is the dual of the Faraday tensor. It is
related to $B^k$ via
\begin{eqnarray}
b^0&=&\frac{WB^kv_k}{\alpha}\,\,,\\
b^i&=&\frac{B^i}{W}+W(B^kv_k)\left(v^i-\frac{\beta^i}{\alpha}\right)\,\,,\\
b^2&=&\frac{B^iB_i}{W^2}+(B^iv_i)^2\,\,.
\end{eqnarray}

Only five of the primitive fluid variables are
independent. The Lorentz factor is defined in terms of the
velocities and the metric as $W = (1-\gamma_{ij}v^i v^j)^{-1/2}$.  
Also one of the pressure, rest-mass density or specific internal energy terms is given in 
terms of the other two by an {\it equation of state}.

The fluxes are usually defined in terms of both the conserved
variables and the primitive variables:
%
\begin{eqnarray}
{\bf F}^i({\bf U}) &=& [D(\alpha v^i - \beta^i), S_j(\alpha v^i -
  \beta^i) + p\delta^i_j, \tau(\alpha v^i - \beta^i) + p
  v^i, \mathcal B^k (\alpha v^i - \beta^i) - \mathcal B^i (\alpha v^k - \beta^k)]^T\ .
\end{eqnarray}
%
The source terms are
%
\begin{eqnarray} \label{source_terms}
{\bf s}({\bf U}) = \Big [0, T^{\mu\nu}\big (\partial_\mu g_{\nu j} +
  \Gamma^\delta_{\mu\nu} g_{\delta j}\big ), \alpha \big (T^{\mu
    0}\partial_\mu \ln \alpha - T^{\mu\nu}\Gamma^0_{\nu\mu} \big), 0 \Big ]\ .
\end{eqnarray}
%
Note that the source terms do not contain differential operators
acting on the stress-energy tensor and that this is important for the
numerical preservation of the hyperbolicity character of the system.
Also note that in a curved spacetime the equations are not in a
strictly-homogeneous conservative form, which is recovered only in flat
spacetime. This conservative form of the relativistic Euler equations
was first derived by the group at the Universidad de Valencia
\cite{Banyuls97} and therefore was named the {\it Valencia
formulation}. 

The stress energy tensor is in turn given by
\begin{eqnarray}
T^{\mu \nu}
  &=&  \left( \rho +  \rho \epsilon + p + b^2 \right) u^\mu u^\nu +
       \left(p + \frac{b^2}{2} \right) g^{\mu \nu} - b^\mu b^\nu
  \\\nonumber
  &\equiv& \rho h^*u^\mu u^\nu + P^* g^{\mu \nu} -
           b^\mu b^\nu.  \label{mhd-stress-energy-tensor}
\end{eqnarray}

%Describe possible equations of state.

%Any other physics points (there'll be lots).

For more detail see the review of Font~\cite{livrevgrrfd} and the GRHydro
paper~\cite{Moesta:2013dna}.

\section{Numerical Implementation}

TODO: describe MHD scheme in particular constrained transport and con2prim
method used.

\section{High Resolution Shock Capturing methods}
\label{sec:hrsc}

The numerical evolution of a hydrodynamical problem is complicated by
the occurrence of {\it shocks}, {\it i.e.} genuine nonlinear discontinuities that
will generically form. It is also complicated by the conservation
constraint. In a High Resolution Shock Capturing (HRSC) method the
requirement of conservation is used to ensure the correct evolution of
a shock. A HRSC method also avoids spurious oscillations at shocks
which are known as Gibbs' phenomena, while retaining a high order of
accuracy over the majority of the domain.

For a full introduction to HRSC methods see~\cite{laney}, \cite{toro},
\cite{leveque}, \cite{livrevsrrfd} and \cite{livrevgrrfd}.

In the {\tt GRHydro} code it was decided to use the {\it method of lines} as a
base for the HRSC scheme. The method of lines is a way of turning a
partial differential equation such as~(\ref{eq:consform1}) into an
ordinary differential equation. For the {\tt GRHydro} code the following steps
are required.
\begin{itemize}
\item Partition the domain of interest into {\it cells}. For
  simplicity we shall assume a regular Cartesian partitioning. This is
  not necessary for the method of lines, but it is for {\tt GRHydro}.
\item Over a given cell with Cartesian coordinates $(x^1_i, x^2_j, x^3_k)$,
  integrate equation~(\ref{eq:consform1}) in space to find the
  ordinary differential equation
  \begin{eqnarray}
    \label{eq:molform1} \nonumber
    \frac{{\rm d} \,}{{\rm d} t} {\bf q} & = & \int \!\!\!\! \int \!\!\!\!
      \int {\bf s} \,{\rm d}^3 x + \int_{x^2_{j-1/2}}^{x^2_{j+1/2}}
    \int_{x^3_{k-1/2}}^{x^3_{k+1/2}} {\bf f}^{(1)} ({\bf q}
    (x^1_{i-1/2}, y, z)) {\rm d} y \, {\rm d} z - \\
    &&    \int_{x^2_{j-1/2}}^{x^2_{j+1/2}} 
    \int_{x^3_{k-1/2}}^{x^3_{k+1/2}} {\bf f}^{(1)} ({\bf q}
    (x^1_{i+1/2}, y, z))
    {\rm d} y \, {\rm d} z + \cdots
  \end{eqnarray}
  where the boundaries of the Cartesian cells are given by $x^1_{i \pm
  1/2}$ and so on.
\item If we define $\bar{\bf q}$ as the integral average of
  ${\bf q}$ over the cell, after dividing (\ref{eq:molform1}) by the volume of the cell, we get an ordinary
  differential equation for $\bar{\bf q}$, in terms of the flux functions and the
  source terms as functions of the spatial differential of $\bar{{\bf
  q}}$. We note that, unlike the spatial differential of ${\bf q}$,
  the spatial differential of $\bar{{\bf q}}$ is well defined in a
  cell containing a discontinuity. 
\end{itemize}

This ordinary differential equation can be solved by the Cactus thorn
MoL. All that {\tt GRHydro} has to do is to return the values of the discrete
spatial differential operator
\begin{eqnarray}
  \label{eq:molrhs1} \nonumber
  {\bf L}({\bf q}) & = & \int \!\!\!\! \int \!\!\!\!
      \int {\bf s} \,{\rm d}^3 x + \int_{x^2_{j-1/2}}^{x^2_{j+1/2}}
    \int_{x^3_{k-1/2}}^{x^3_{k+1/2}} {\bf f}^{(1)} ({\bf q}
    (x^1_{i-1/2}, y, z)) {\rm d} y \, {\rm d} z - \\
    &&    \int_{x^2_{j-1/2}}^{x^2_{j+1/2}} 
    \int_{x^3_{k-1/2}}^{x^3_{k+1/2}} {\bf f}^{(1)} ({\bf q}
    (x^1_{i+1/2}, y, z)) {\rm d} y \, {\rm d} z + \cdots
\end{eqnarray}
given the data ${\bf q}$, and to supply the boundary conditions that will
be required to calculate this right hand side at the next time level.
We note that in the current implementation of MoL the solution to the
ordinary differential equation~(\ref{eq:molform1}) will be $N^{\rm
  th}$-order accurate provided that the time integrator used by MoL is $N^{\rm
  th}$-order accurate in time, and that the discrete operator ${\bf L}$ is
$N^{\rm th}$-order accurate in space and {\it first}-order (or better)
accurate in time.  For more details on the method of lines, and the
options given with the time integration for MoL, see the relevant
chapter in the ThornGuide.

In this implementation of {\tt GRHydro} the right hand side operator ${\bf L}$
will be simplified considerably by approximating the integrals by the
midpoint rule to get
\begin{equation}
  \label{eq:molrhs2}
  {\bf L}({\bf q}) = {\bf s}_{i,j,k} + {\bf f}^{(1)}_{i-1/2,j,k} -
    {\bf f}^{(1)}_{i+1/2,j,k} + \cdots
\end{equation}
where the dependence of the flux on ${\bf q}$ and spatial position is
implicit in the notation. Given this simplification, the calculation
of the right hand side operator splits simply into the following two parts:
\begin{enumerate}
\item Calculate the source terms ${\bf s}({\bf q}(x^1_i, x^2_j,
  x^3_k))$:

  Given that $\bar{{\bf q}}$ is a second-order accurate approximation
  to ${\bf q}$ at the midpoint of the cell, which is precisely $(x^1_i, x^2_j,
  x^3_k)$, for second-order accuracy it is sufficient to use ${\bf
  s}(\bar{{\bf q}}_{i,j,k})$.
\item In each coordinate direction $x^a$, calculate the {\it intercell
    flux} ${\bf f}^{(a)}({\bf q}_{i+1/2,j,k})$:
  
  From the initial data $\bar{{\bf q}}$ given at time $t^n$ we can
  reconstruct the data at the cell boundary, $({\bf q}_{i+1/2,j,k})$
  to any required accuracy in space (except in the vicinity of a
  shock, where only first-order accuracy is guaranteed).
% - FIXME: check,
%  Ian thinks the ENO theorems say that for linear problems get high order
%  even with discty). 
  However this will only be zeroth-order accurate
  in time. To ensure first-order accuracy in time, we have to find
  $({\bf q}_{i+1/2,j,k})(t)$ while retaining the high spatial order
  of accuracy. This requires two steps:
  \begin{enumerate}
  \item {\it Reconstruct} the data ${\bf q}$ over the cells adjacent
    to the required cell boundary. This reconstruction should use the
    high spatial order of accuracy. This gives two values of
    $({\bf q}_{i+1/2,j,k})$, which we call ${\bf q}_L$ and ${\bf q}_R$,
    where ${\bf q}_L$ is obtained from cell $i$ (left cell) and ${\bf q}_R$
    from cell $i+1$ (right cell).
  \item The values ${\bf q}_{L,R}$ are used as initial data for the
    {\it Riemann problem}. This is the initial value problem given by
    the partial differential equation~(\ref{eq:consform1}) with
    semi-infinite piecewise constant initial data ${\bf q}_{L,R}$. As
    the true function ${\bf q}$ is probably not piecewise constant we
    will not get the exact solution of the general problem even if we
    solve the local Riemann problem exactly. However, it will be first-order 
    accurate in time and retain the high order of accuracy in
    space which, as explained in the documentation to the MoL thorn,
    is sufficient to ensure that the method as a whole has a high
    order of accuracy. 
  \end{enumerate}
\end{enumerate}

So, the difficult part of {\tt GRHydro} is expressed in two routines. One that
reconstructs the function ${\bf q}$ at the boundaries of a
computational cell given the cell average data $\bar{{\bf q}}$, and
another that calculates the intercell flux ${\bf f}$ at this cell
boundary. 

\section{Reconstruction}
\label{sec:recon}

In the reduction of all of {\tt GRHydro} to two routines in the last section
one point was glossed over. That is, in order for the numerical method
to be consistent and convergent it must retain conservation and not
introduce spurious oscillations. Up to this point all the steps have
either been exact or have neither changed the conservation properties
or the profile of the function. Also, the calculation of the intercell
flux from the Riemann problem can be made to be ``exactly correct''.
That is, even though as explained above it may not be the true flux
for the real function ${\bf q}$, it will be the exact physical
solution for the values ${\bf q}_{L,R}$ given by the reconstruction
routine, so the intercell flux cannot violate conservation or
introduce oscillations.  Unphysical effects such as these can only be
introduced by an incorrect reconstruction.

For a full explanation of reconstruction methods see
Laney~\cite{laney}, Toro~\cite{toro}, Leveque~\cite{leveque}. For the
moment we will concentrate on the simplest methods that are better
than first-order accurate in space, the TVD slope-limited methods.
More complex methods such as ENO will be introduced later.

In the late 1950's Godunov proved a theorem that, in this context,
says
\begin{quote}
Any {\bf linear} reconstruction method of higher-than-first-order
accuracy may introduce spurious oscillations.
\end{quote}
For this theorem {\it linear} meant that the reconstruction method was
independent of the data it was reconstructing. Simple centred
differencing is a linear second-order method and is oscillatory near a
shock. Instead we must find a reconstruction method that depends on
the data ${\bf q}$ being reconstructed.

Switching our attention to conservation, we note that there is
precisely one conservative first-order reconstruction method,
\begin{equation}
  \label{eq:reconfirst}
  {\bf q}^{{\rm First}}(x) = \bar{{\bf q}}_i, \qquad x \in [
  x_{i-1/2}, x_{i+1/2} ], 
\end{equation}
and that any second-order conservative method can be written in terms
of a {\it slope} or rather \emph{difference} $\Delta_i$ as
\begin{equation}
  \label{eq:reconsecond}
  {\bf q}^{{\rm Second}}(x) = \bar{{\bf q}}_i + \frac{x -
  x_i}{x_{i+1/2} - x_{i-1/2}} \Delta_i, \qquad x \in [ x_{i-1/2}, x_{i+1/2} ].
\end{equation}

\subsection{TVD Reconstruction}
\label{sec:tvd}

As we want a method that is accurate ({\it i.e.}, at least to second order)
while being stable ({\it i.e.}, only first order or nonlinear at shocks)
then the obvious thing to do is to use some second-order method in the
form of equation~(\ref{eq:reconsecond}) in smooth regions but which
changes to the form of equation~(\ref{eq:reconfirst}) smoothly near
shocks. 

In the articles describing the {\tt GRAstro\_Hydro}
code\footnote{http://wugrav.wustl.edu/ASC/internal/asccodes.html},
this was described as an average of the two reconstructions, 
\begin{equation}
  {\bf q}^{{\rm TVD}}(x) = \phi({\bf q}) {\bf q}^{{\rm Second}} + (1 -
  \phi({\bf q})) {\bf q}^{{\rm First}},
  \label{First_qTVD}
\end{equation}
where $\phi \in [0,1]$ varies smoothly in some sense, and is zero near
a shock and 1 in smooth regions. In Toro's notation~\cite{toro} (which we usually adopt here) 
the slope limiter function $\phi$ (having the same attributes as above)
directly multiplies the slope, giving
\begin{equation}
  {\bf q}^{{\rm TVD}}(x) = \bar{{\bf q}}_i + \frac{x -
  x_i}{x_{i+1/2} - x_{i-1/2}} \phi({\bf q}) \Delta_i, \qquad x \in [
  x_{i-1/2}, x_{i+1/2} ]. 
  \label{Toro_qTVD}
\end{equation}
Equations (\ref{First_qTVD}) and (\ref{Toro_qTVD}) are equivalent.

For details on how to construct a limiter, on their stability regions and on 
the explicit expressions for the limiters used here,
see~\cite{toro}. The {\tt GRHydro} code implements the {\tt minmod} limiter
(the most diffusive and the default), the Van Leer Monotonized Centred
(MC) ({\tt VanLeerMC}) limiter in a number of forms (which should give equivalent results), 
and the {\tt Superbee} limiter. The limiter specified by the parameter value {\tt VanLeerMC2} 
is the recommended one.

As an example, we show how TVD with the minmod limiter is implemented
in the code. First, we define the minmod function:
\begin{equation}
  \label{eq:tvdminmod}
  \mathrm{\mathbf{minmod}}(a,b) = \left\{ \begin{array}{c l} 
      \text{min}(|a|,|b|) & \text{if } (a b > 0)\\
      0 & \text{otherwise}. \end{array}\right.
\end{equation}
For reconstructing $\mathbf{q}$ %at the cell interfaces $i+\frac{1}{2}$ and $i-\frac{1}{2}$, 
we choose two differences
\begin{equation}
  \begin{array}{lcl}
    % tex4ht fails unless \mathbf is in a group
    \Delta_{\mathrm{upw}} & = & {\mathbf{q}}_{i} - {\mathbf{q}}_{i-1}\\
    \Delta_{\mathrm{loc}} & = & {\mathbf{q}}_{i+1} - {\mathbf{q}}_{i}\,,\\
  \end{array}
\end{equation}
and write
\begin{equation}
  {\bf q}^{{\rm TVD,minmod}}(x) = \bar{{\bf q}}_i + \frac{x -
  x_i}{x_{i+1/2} - x_{i-1/2}} \mathbf{minmod}(\Delta_{\mathrm{upw}},\Delta_{\mathrm{loc}}), 
    \qquad x \in [
  x_{i-1/2}, x_{i+1/2} ]. 
\end{equation}
\subsection{PPM reconstruction}
\label{sec:ppm}

The piecewise parabolic method (PPM) of Colella and
Woodward~\cite{ppm} is a rather more complex method that requires a
number of steps. The implementation in the {\tt GRHydro} code is specialized
to use evenly spaced grids. Also, some of the more complex features are not
implemented; in particular, the dissipation algorithm is only the
simplest given in the original article. Here we just give the implementation
details. For more details on the method we refer to the original
article.

Again we assume we are reconstructing a scalar function $q$ as a
function of $x$ in one dimension on an evenly spaced grid, with spacing $\Delta
x$. The first step is to interpolate a quadratic polynomial to the cell
boundary, 
\begin{equation}
  \label{eq:ppm1}
  q_{i+1/2} = \frac{1}{2} \left( q_{i+1} + q_i \right) + \frac{1}{6}
  \left( \delta_m q_i - \delta_m q_{i+1} \right), 
\end{equation}
where
\begin{equation}
  \label{eq:ppmdm1}
  \delta_m q_i = \left\{ \begin{array}{c l} \text{min}(|\delta q_i|,
      2|q_{i+1} - q_i|, 2|q_i - q_{i-1}|) \text{ sign}(\delta q_i) &
      \text{if } (q_{i+1} - q_i)(q_i - q_{i-1}) > 0, \\
      0 & \text{otherwise}. \end{array} \right.,
\end{equation}
and
\begin{equation}
  \label{eq:ppmd1}
  \delta q_i = \frac{1}{2}(q_{i+1} - q_{i-1}).
\end{equation}
At this point we set both left and right states at the interface to be
equal to this,
\begin{equation}
  \label{eq:ppmset1}
  q_i^R = q_{i+1}^L = q_{1+1/2}.
\end{equation}

This reconstruction will be oscillatory near shocks. To preserve
monotonicity, the following replacements are made:
\begin{eqnarray}
  \label{eq:ppmmonot}
  q_i^L = q_i^R = q_i & \text{if} & (q_i^R - q_i)(q_i - q_i^L) \leq
  0 \\
  q_i^L = 3 q_i - 2q_i^R & \text{if} & (q_i^R - q_i^L)\left( q_i -
  \frac{1}{2} (q_i^L + q_i^R) \right) > \frac{1}{6}(q_i^R - q_i^L)^2
  \\ 
  q_i^R = 3 q_i - 2q_i^L & \text{if} & (q_i^R - q_i^L)\left( q_i -
  \frac{1}{2} (q_i^L + q_i^R) \right) < -\frac{1}{6}(q_i^R -
  q_i^L)^2. 
\end{eqnarray}

However, before applying the monotonicity preservation two other steps
may be applied. Firstly we may steepen discontinuities. This is to
ensure sharp profiles and is only applied to contact
discontinuities. This may be switched on or off using the parameter
{\tt ppm\_detect}. This part of the method replaces the cell boundary
reconstructions of the density with
\begin{eqnarray}
  \label{eq:ppmdetect}
  \rho_i^L & = & \rho_i^L (1-\eta) + \left(\rho_{i-1} + \frac{1}{2}
    \delta_m \rho_{i-1} \right) \eta \\
  \rho_i^R & = & \rho_i^R (1-\eta) + \left(\rho_{i+1} - \frac{1}{2}
    \delta_m \rho_{i+1} \right) \eta
\end{eqnarray}
where $\eta$ is only applied if the discontinuity is mostly a contact
(see~\cite{ppm} for the details) and is defined as
\begin{equation}
  \label{eq:ppmeta}
  \eta = \text{max}(0, \text{min}(1, \eta_1 (\tilde{\eta} - \eta_2))),
\end{equation}
where $\eta_1,\eta_2$ are positive constants and
\begin{equation}
  \label{eq:ppmetatilde}
  \tilde{\eta} = \left\{ \begin{array}{c l} 
  \frac{\rho_{i-2} - \rho_{i+2} + 4 \delta\rho_i}{12\delta\rho_i} &
  \text{ if } \left\{
\begin{array}{l}
  \delta^2\rho_{i+1}\delta^2\rho_{i-1} < 0\\
  (\rho_{i+1} - \rho_{i-1}) - \epsilon \text{min}(|\rho_{i+1}|,|\rho_{i-1}|) >
  0\nonumber
\end{array} \right. \\
  0 & \text{otherwise} \end{array} \right.,
\end{equation}
with $\epsilon$ another positive constant and
\begin{equation}
  \label{eq:ppmd2rho}
  \delta^2\rho_i = \frac{\rho_{i+1} - 2\rho_i + \rho_{i-1}}{6\Delta
    x^2}.  
\end{equation}

Another step that is performed before monotonicity enforcement is to
flatten the zone structure near shocks. This adds simple dissipation
and is always in the code. In short, the reconstructions are again
altered to
\begin{equation}
  \label{eq:ppmflatten}
  q_i^{L,R} = \nu_i q_i^{L,R} + (1 - \nu_i) q_i,
\end{equation}
where
\begin{equation}
  \label{eq:ppmflatten2}
  \nu_i = \left\{ \begin{array}{c l} {\rm max}(0, 1 - \text{max}(0, \omega_2
      (\frac{p_{i+1} - p_{i-1}}{p_{i+2} - p_{i-2}} - \omega_1))) & \text{
        if } \omega_0 \text{min}(p_{i-1}, p_{i+1}) < |p_{i+1} - p_{i - 1}|
      \text{ and } v^x_{i-1} - v^x_{i+1} > 0 \\
      1 & \text{otherwise} \end{array} \right.
\end{equation}
and $\omega_0, \omega_1,\omega_2$ are constants.

The above flattening procedure is not the one in the original article of Colella and Woodward, but
it has been adapted from it in order to have a stencil of three points.  The original flattening
procedure is also implemented in {\tt GRHydro}. Instead of \ref{eq:ppmflatten}, it consists in the formula
\begin{equation}
  \label{eq:ppmflatten-stencil4}
  q_i^{L,R} = \tilde \nu_i q_i^{L,R} + (1 - \tilde \nu_i) q_i,
\end{equation}
where
\begin{eqnarray}
\tilde \nu_i &=& {\rm max}\Big(\nu_i,\nu_{i+{\rm sign}(p_{i-1}-p_{i+1})}\Big)
\end{eqnarray}
and $\nu_i$ is given by \ref{eq:ppmflatten2}. This can be activated by setting the parameter {\tt
ppm\_flatten} to {\tt stencil\_4}. Formula \ref{eq:ppmflatten-stencil4}, despite requiring
more computational resources (especially when mesh refinement is used), usually gives very similar
results to \ref{eq:ppmflatten}, so we routinely use \ref{eq:ppmflatten}.


\subsection{ENO Reconstruction}
\label{sec:eno}

An alternative way of getting higher-than-second-order accuracy is the implementation of the
Essentially Non-Oscillatory methods of Harten et.al~\cite{eno}. The
essential idea is to alter the stencil to use those points giving the
smoothest reconstruction. The only restriction is that the stencil
must include the cell to be reconstructed (for stability). Here we
describe the simplest ENO type reconstruction: piecewise polynomial
reconstruction using the (un)divided differences to measure the
smoothness. 

Let $k$ be the order of the reconstruction. Suppose we are
reconstructing the scalar function $q$ in cell $i$. We start with the
cell $I_i$. We then add to the stencil cell $I_j$, where $j = i \pm
1$, where we choose $j$ to minimize the Newton divided differences
\begin{eqnarray}
  \label{enodd}
  q \left[ x_{i-1}, x_i \right] & = & \frac{q_i - q_{i-1}}{x_{i+1/2}
    - x_{i-3/2}} \\
  q \left[ x_i, x_{i+1} \right] & = & \frac{q_{i+1} - q_i}{x_{i+3/2}
    - x_{i-1/2}}.
\end{eqnarray}
\noindent We then recursively add more cells, minimizing the higher-order 
Newton divided differences $q \left[ x_{i-k}, \dots, x_{i+j} \right]$
defined by 
\begin{equation}
  \label{enodd2}
  q \left[ x_{i-k}, \dots, x_{i+j} \right] = \frac{ q \left[ x_{i-k+1},
  \dots, x_{i+j} \right] - q \left[ x_{i-k}, \dots, x_{i+j-1} \right]
  }{x_{i+j} - x_{i-k}}.
\end{equation}
\noindent The reconstruction at the cell boundary is given by a
standard $k^{\text{th}}$-order polynomial interpolation on the chosen
stencil. 

\cite{shueno} has outlined an elegant way of calculating the cell
boundary values solely in terms of the stencil and the known data. If
the stencil is given by
\begin{equation}
  \label{enostencil1}
  S(i) = \left\{ I_{i-r}, \dots, I_{i+k-r-1} \right\},
\end{equation}
\noindent for some integer $r$, then there exist constants $c_{rj}$
depending only on the grid $x_i$ such that the boundary
values for cell $I_i$ are given by
\begin{equation}
  \label{enoc1}
  q_{i+1/2} = \sum_{j=0}^{k-1} c_{rj} q_{i-r+j}, \qquad q_{i-1/2} =
  \sum_{j=0}^{k-1} c_{r-1,j} q_{i-r+j}. 
\end{equation}
\noindent The constants $c_{rj}$ are given by the rather complicated
formula 
\begin{equation}
  \label{enoc2}
  c_{rj} = \left\{ \sum_{m=j+1}^k \frac{ \sum_{l=0, l \neq m}^k
  \prod_{q=0, q \neq m, l}^k \left(  x_{i+1/2} - x_{i-r+q-1/2} \right)
  }{ \prod_{l=0, l \neq m}^k \left(  x_{i-r+m-1/2} - x_{i-r+L-1/2}
  \right) }  \right\} \Delta x_{i-r+j}. 
\end{equation}
\noindent This simplifies considerably if the grid is even. The
coefficients for an even grid are given (up to seventh order)
by~\cite{shueno}.

\section{Riemann Problems}
\label{sec:riemann}

Given the reconstructed data, we then solve a local Riemann problem in
order to get the intercell flux. The Riemann problem is specified by
an equation in flux conservative homogeneous form,
\begin{equation}
  \label{eq:homconsform1}
  \partial_t {\bf q} + \partial_{x^i} {\bf f}^{(i)} ({\bf q}) = 0
\end{equation}
with piecewise constant initial data ${\bf q}_{_{L,R}}$ separated by a
discontinuity at $x^{(1)}=0$. Flux terms for the other directions are
given similarly.  There is no intrinsic scale to this problem and so
the solution must be a self similar solution with similarity variable
$\xi = x^{(1)}/t$. The solution is given in terms of {\it waves} which
separate different {\it states}, with each state being constant. The
waves are either {\it shocks}, across which all hydrodynamical
variables change discontinuously, {\it rarefactions}, across which all
the variables change continuously (the wave is not a single value of
$\xi$ for a rarefaction, but spreads across a finite range), or {\it
contact or tangential} discontinuities, across which some but not all
of the hydrodynamical variables change discontinuously and the rest
are constant. The characteristics of the matter evolution converge and
break at a shock, diverge at a rarefaction and are parallel at the
other linear discontinuities.

The best references for solving the Riemann problem either exactly or
approximately are~\cite{leveque}, \cite{toro}, \cite{laney}. Here, we start by 
giving a simple outline. We
start by considering the $N$ dimensional linear problem in one
dimension given by
\begin{equation}
%  \label{eq:linsys1}
  \label{lsrp}
  \partial_t {\bf q} + A \partial_{x} {\bf q} = 0  \ ,
\end{equation}
where $A$ is a $N\times N$ matrix with constant coefficients. We
define the eigenvalues $\lambda^j$ with associated right and left
eigenvectors ${\bf r}^j,{\bf l}_j$, where the eigenvectors are
normalized to each other ({\it i.e.}, their dot product is
$\delta^i_j$). 
%%%%This was sort of written twice
%Defining the characteristic variables ${\bf w}_i$ as
%\begin{equation}
%  \label{eq:charvar1}
%  {\bf w}_i = {\bf l}^j_i \cdot {\bf q}_j \ ,
%\end{equation}
%then equation (\ref{eq:linsys1}) transforms to a set of uncoupled
%linear equations
%\begin{equation}
%  \label{eq:linsys2}
%  \partial_t {\bf w}_i + \Lambda \partial_{x} {\bf w}_i = 0   \ ,
%\end{equation}
%where the matrix $\Lambda$ contains the eigenvalues on the diagonals
%and is zero elsewhere. By convention the eigenvalues are ordered
%$\lambda_1 < \dots < \lambda_N$.
%
%The solution to one of the uncoupled equations along the cell boundary
%is simply that the characteristic variable is given by
%\begin{equation}
%  \label{eq:linflux}
%  {\bf w}_i = \left\{ \begin{array}[c]{r c l} ({\bf w}_i)_L & {\rm if}
%      & \lambda_i > 0 \\ 
%      ({\bf w}_i)_R & {\rm if} & \lambda_i < 0. \end{array}\right\},
%\end{equation}
%where, as before, subscripts $L$ and $R$ are used to denote quantities
%obtained from cells on the left and right, respectively, of the intercell 
%interface.
%To get the intercell flux we transform the solution back to the
%conserved variables by taking the dot product with the right
%eigenvectors and then multiplying by the matrix $A$. This can be written
%in a number of ways; the form we use is
%\begin{equation}
%  \label{eq:linsysflux}
%  {\bf f}_{i+1/2} = \frac{1}{2} \left[ A({\bf q}_{_L}) + A({\bf q}_{_R})
%  - \Sigma_{j=1}^N |\lambda_j| \Delta {\bf w}_j {\bf r} \right], 
%\end{equation}
%where $\Delta {\bf w}^j$ is the jump in the appropriate characteristic
%variable across the $j^{{\rm th}}$ wave. We note that the
%characteristic jumps are easily calculated from
%\begin{equation}
%  \label{eq:charjump}
%  \Delta {\bf w}_i = {\bf l}_i^j \left[ ({\bf q}_R)_j - ({\bf q}_L)_j
%  \right]. 
%\end{equation}
%
%The Roe flux is simply given by locally assuming that the conserved
%variables are constant. Thus the matrix $A$ is simply given by
%evaluating the Jacobian matrix $\partial {\bf f}({\bf q}) / \partial
%{\bf q}$ at some average state. Equation~(\ref{eq:linsysflux}) is then
%used to evaluate the intercell flux. The only questions are
%\begin{itemize}
%\item What is an appropriate intermediate state?
%\item When is this approximate (in)valid?
%\end{itemize}
%
%\subsection{Linear system}
%%
%Let ${\bf l}_i$, ${\bf r}^i$ be the left and right eigenvectors,
%respectively, of the matrix $A$ with eigenvalues $\lambda_i$,
%normalized such that ${\bf l}_i \cdot {\bf r}^j = \delta_i^j$. 
%
\label{sec:linsys}
We shall assume that the eigenvectors span the space. The characteristic
variables ${\bf w}_i$ are defined by
\begin{equation}
  \label{charvar}
  {\bf w}_i = {\bf l}_i \cdot {\bf q}.
\end{equation}
\noindent Then equation (\ref{lsrp}) when written in terms of the
characteristic variables becomes
\begin{equation}
  \label{charvarrp}
  \partial_t {\bf w} + \Lambda \partial_x {\bf w} = 0,
\end{equation}
\noindent where $\Lambda$ is the matrix containing the eigenvalues
$\lambda_i$ on the diagonals and zeros elsewhere. Hence each
characteristic variable ${\bf w}^i$ obeys the linear advection
equation with velocity $a = \lambda_i$. This solves the Riemann
problem in terms of characteristic variables.

In order to write the solution in terms of the original variables
${\bf q}$ we order the variables in such a way  that $\lambda_1 \leq \dots \leq
\lambda_N$. We also define the differences in the characteristic
variables $\Delta {\bf w}_i = ({\bf w}_i)_L - ({\bf w}_i)_R$ across
the $i^{{\rm th}}$ characteristic wave. These differences are single
numbers (`scalars'). We note that these differences can easily be
found from the initial data using
\begin{equation}
  \label{dw}
  \Delta {\bf w}_i = {\bf l}_i \cdot \left( {\bf q}_L - {\bf q}_R
  \right). 
\end{equation}
\noindent As the change in the solution across each wave is precisely
the difference in the associated characteristic variable, the solution
of the Riemann problem in terms of characteristic variables can be
written as either
\begin{equation}
  \label{lsrpsol1}
  {\bf w}_i = ({\bf w}_i)_L + \sum_{j=1}^M \Delta {\bf w}_j {\bf e}^j \quad
  {\rm if}\ \lambda_M < \xi < \lambda_{M+1},
\end{equation}
\noindent or 
\begin{equation}
  \label{lsrpsol2}
  {\bf w} = ({\bf w}_i)_R - \sum_{j=M+1}^N \Delta {\bf w}_j {\bf e}^j
  \quad {\rm if}\ \lambda_M < \xi < \lambda_{M+1},
\end{equation}
\noindent or as the average
\begin{equation}
  \label{lsrpsol3}
  {\bf w}_i = \frac{1}{2} \left( ({\bf w}_i)_L + ({\bf w}_i)_R +
  \sum_{j=1}^M \Delta 
  {\bf w}_j {\bf e}^j - \sum_{j=M+1}^N \Delta {\bf w}_j {\bf e}^j 
  \right) \quad {\rm if}\ \lambda_M < \xi < \lambda_{M+1},
\end{equation}
\noindent where ${\bf e}^i$ is the column vector $({\bf e}^i)_j =
\delta^i_j$. 

Converting back to the original variables ${\bf q}$ we have the
solution
\begin{equation}
  \label{lsrpsol4}
  {\bf q} = \frac{1}{2} \left( {\bf q}_L + {\bf q}_R + \sum_{i=1}^M \Delta
  {\bf w}_i {\bf r}^i - \sum_{i=M+1}^N \Delta {\bf w}_i {\bf r}^i
  \right) \quad {\rm if}\ \lambda_M < \xi < \lambda_{M+1}.
\end{equation} 
\noindent In the case where we are only interested in the flux
along the characteristic $\xi = 0$ we can write the solution in the
simple form
\begin{equation}
  \label{lsrpsol6}
  {\bf f}({\bf q}) = \frac{1}{2} \left( {\bf f}({\bf q}_L) + {\bf f}({\bf
  q}_R) - \sum_{i=1}^N | \lambda_i | \Delta {\bf w}_i {\bf r}^i \right).
\end{equation} 

All exact Riemann solvers have to solve at least an implicit
equation and so are computationally very expensive. As the 
solution of Riemann problems takes a large portion of
the time to run in a HRSC code, approximations that speed the
calculation of the intercell flux are often essential. This is
especially true in higher dimensions (>1), where the solution of the
ordinary differential equation to give the relation across a
rarefaction wave makes the use of an exact Riemann solver impractical.

Approximate Riemann solvers can have problems, as shown in depth by
Quirk~\cite{Quirk}. Hence it is best to compare the
results of as many different solvers as possible. Here we shall describe
the three approximate solvers used in this code, starting with the
simplest.

\subsection{HLLE solver}
\label{sec:hlle}

The Harten-Lax-van Leer-Einfeldt (HLLE) solver of Einfeldt~\cite{Einfeldt88} is a
simple two-wave approximation. We assume that the maximum and minimum wave speeds
$\xi_{\pm}$ are known. The solution of the Riemann problem is then
given by requiring conservation to hold across the waves. The solution
takes the form
\begin{equation}
  \label{hlle1}
  {\bf q} = \left\{ \begin{array}[c]{r c l} {\bf q}_L & {\rm if} & \xi
        < \xi_- \\  {\bf  q}_* & {\rm if} & \xi_- < \xi < \xi_+ \\
        {\bf q}_R & {\rm if} & \xi   > \xi_+, \end{array}\right. 
\end{equation}
\noindent and the intermediate state ${\bf q}_*$ is given by
\begin{equation}
  \label{hlle2}
  {\bf q}_* = \frac{\xi_+ {\bf q}_R - \xi_- {\bf q}_L - {\bf f}({\bf
  q}_R) + {\bf f}({\bf q}_L)}{\xi_+ - \xi_-}.
\end{equation}
\noindent If we just want the numerical flux along the boundary then
this takes the form
\begin{equation}
  \label{hlleflux}
  {\bf f}({\bf q}) = \frac{\widehat{\xi}_+{\bf f}({\bf q}_L) -
  \widehat{\xi}_-{\bf f}({\bf q}_R) + \widehat{\xi}_+ \widehat{\xi}_-
  ({\bf q}_R - {\bf q}_L)}{\widehat{\xi}_+ - \widehat{\xi}_-},
\end{equation}
\noindent where
\begin{equation}
  \label{hlle3}
  \widehat{\xi}_- = {\rm min}(0, \xi_-), \quad \widehat{\xi}_+ =
  {\rm max}(0, \xi_+). 
\end{equation}

Knowledge of the precise minimum and maximum characteristic velocities
$\xi_{\pm}$ requires knowing the solution of the Riemann problem.
Instead, the characteristic velocities are usually found from the
eigenvalues of the Jacobian matrix $\partial {\bf f} / \partial {\bf
  q}$ evaluated at some intermediate state. To ensure that the maximum
and minimum eigenvalues over the entire range between the left and
right states are found, we evaluate the Jacobian in both the left and
right states and take the maximum and minimum over all eigenvalues.
This ensures, for the systems of equations considered here, that the
real maximum and minimum characteristic velocities are contained
within $[\xi_-, \xi_+]$.

If we set $\alpha = {\rm max}(|\xi_-|, |\xi_+|)$ and replace the
characteristic velocities $\xi_{\pm}$ with $\pm \alpha$, we find the
Lax--Friedrichs flux ({\it cf.} also Tadmor's semi-discrete scheme~\cite{Tadmor00})
\begin{equation}
  \label{lfflux}
  {\bf f}({\bf q}) = \frac{1}{2} \left[ {\bf f}({\bf q}_L) + {\bf f}({\bf
  q}_R) + \alpha ({\bf q}_L -{\bf q}_R) \right].
\end{equation}
\noindent This is very diffusive, but also very stable.


\subsection{Roe solver}
\label{sec:roe}

The linearized solver of Roe~\cite{Roe81} is probably the most popular
approximate Riemann solver. The simplest interpretation is that the
Jacobian $\partial {\bf f} / \partial {\bf q}$ is linearized about
some intermediate state. Then the conservation form reduces to the
linear equation
\begin{equation}
  \label{roe1}
  \partial_t {\bf q} + A \partial_x {\bf q} = 0,
\end{equation}
\noindent where $A$ is a constant coefficient matrix. This is
identical to equation (\ref{lsrp}) and so all the results of
section~\ref{sec:linsys} on linear systems hold. We reiterate that the
standard form for the flux along the characteristic ray $\xi=0$ is
\begin{equation}
  \label{roe2}
  {\bf f}({\bf q}) = \frac{1}{2} \left( {\bf f}({\bf q}_L) + {\bf f}({\bf
  q}_R) - \sum_{i=1}^N | \lambda_i | \Delta {\bf w}_i {\bf r}^i \right).
\end{equation} 

There is a choice of which intermediate state the Jacobian should be
evaluated at. Roe gives three criteria that ensure the consistency and
stability of the numerical flux:
\begin{enumerate}
\item $A({\bf q}_{{\rm Roe}}) \left( {\bf q}_R - {\bf q}_L \right) =
  {\bf f}({\bf q}_R) - {\bf f}({\bf q}_L)$,
\item $A({\bf q}_{{\rm Roe}})$ is diagonalizable with real
  eigenvalues, 
\item $A({\bf q}_{{\rm Roe}}) \rightarrow \partial {\bf f} / \partial
  {\bf q}$ smoothly as ${\bf q}_{{\rm Roe}} \rightarrow {\bf q}$.
\end{enumerate}
\noindent A true Roe average for relativistic hydrodynamics, {\it i.e.}, an
intermediate state that satisfies all these conditions, has been
constructed by Eulderink~\cite{Eulderink94}. However, frequently it is sufficient
to use
\begin{equation}
  \label{roe3}
  {\bf q}_{{\rm Roe}} = \frac{1}{2} \left( {\bf q}_R + {\bf q}_L \right),
\end{equation}
\noindent which satisfies only the last two conditions. For simplicity
we have implemented this arithmetic average.


\subsection{Marquina solver}
\label{sec:marq}

Unlike all the other Riemann solvers introduced so far, the Marquina
solver as outlined in \cite{Marquina1} does not solve the Riemann
problem completely. Instead, only the flux along the characteristic
ray $\xi=0$ is given. It can be seen as a generalized Roe solver, as
the results are the same except at sonic points. These points are
where the fluid velocity is equal to the speed of sound of the fluid.
In the context of Riemann problems, sonic points are found when the
ray $\xi=0$ is within a rarefaction wave.

Firstly define the left ${\bf l}({\bf q}_{L,R})$ and right ${\bf
  r}({\bf q}_{L,R})$ eigenvectors and the eigenvalues $\lambda({\bf
  q}_{L,R})$ of the Jacobian matrix $\partial {\bf f} / \partial {\bf
  q}$ evaluated at the left and right states. Next define left and
right characteristic variables ${\bf w}_{L,R}$ and fluxes
${\bf \phi}_{L,R}$ by
\begin{equation}
  \label{marq1}
  ({\bf w}_i)_{L,R} = {\bf l}_i({\bf q}_{L,R}) \cdot {\bf q}_{L,R}, \quad
  ({\bf \phi}_i)_{L,R} = {\bf l}_i({\bf q}_{L,R}) \cdot {\bf f}({\bf
  q}_{L,R}). 
\end{equation}

Then the algorithm chooses the correct-sided characteristic flux if
the eigenvalues $\lambda_i({\bf q}_L)$, $\lambda_i({\bf q}_R)$ have
the same sign, and uses a Lax--Friedrichs type flux if they change
sign. In full, the algorithm is given in figure~\ref{fig:marqcode}.
\begin{figure}[htbp]
  \begin{center}
    \leavevmode
\begin{equation}
  \label{marqalg}
  \begin{array}[l]{l}
    {\bf For}\ \, i = 1, \dots, N\ {\bf do} \\
    \qquad \begin{array}[c]{l}
      {\bf If}\ \, \lambda_i({\bf q}_L) \lambda_i({\bf q}_R) > 0
        \ \, {\bf then} \\
      \qquad \qquad \begin{array}[c]{l}
        {\bf If}\ \,  \lambda_i({\bf q}_L) > 0 \ \, {\bf then} \\
        \qquad \qquad \qquad \begin{array}[c]{r c l}
          {\bf \phi}^i_+ & = & {\bf \phi}^i_L \\
          {\bf \phi}^i_- & = & 0
        \end{array} \\
        {\bf else}  \\
        \qquad \qquad \qquad \begin{array}[c]{r c l}
          {\rm \phi}^i_+ & = & 0 \\
          {\rm \phi}^i_- & = & {\bf \phi}^i_R
        \end{array} \\
        {\bf end if}
      \end{array} \\
      {\bf else} \\
      \qquad \qquad \begin{array}[c]{r c l}
        \alpha^i & = & {\rm max}(|\lambda_i({\bf q}_L), \lambda_i({\bf
          q}_R)|) \\
        {\bf \phi}^i_+ & = & \frac{1}{2} \left( {\bf \phi}^i_L +
          \alpha^i {\bf w}^i_L \right) \\
        {\bf \phi}^i_- & = & \frac{1}{2} \left( {\bf \phi}^i_R -
          \alpha^i {\bf w}^i_R \right)
      \end{array} \\
      {\bf end if} \\
    \end{array} \\
    {\bf end do}
  \end{array}
\end{equation}
    \caption[The algorithm to calculate the Marquina flux]{The
      algorithm to calculate the Marquina flux.}
    \label{fig:marqcode}
  \end{center}
\end{figure}

Then the numerical flux is given by
\begin{equation}
  \label{marqflux}
  {\bf f}({\bf q}) = \sum_{i=1}^N \left[ {\bf \phi}^i_+ 
  {\bf r}^i ({\bf q}_L) + {\bf \phi}^i_- {\bf r}^i ({\bf q}_R)
  \right].
\end{equation}

The above implementation is based on \cite{Aloy99b}.


\section{Other points in {\tt GRHydro}}
\label{sec:misc}

There are a number of other things done by {\tt GRHydro} which, whilst not as
important as reconstruction and evolution, are still essential.


\subsection{Source terms}
\label{sec:sources}

In a curved spacetime the equations are not in homogeneous conservation-law 
form but also contain source terms. These are actually calculated
first, before the flux terms (it simplifies the loop very slightly).
There are a few points to note about the calculation of the sources.
\begin{itemize}
\item The calculation used here, taken from the {\tt GR3D} code, requires both the
  metric and the extrinsic curvature.
\item In order to calculate the Christoffel symbols the gauge and
  metric variables must be differenced. Currently centred differencing
  of second or fourth (we are safe to use this, as {\tt GRHydro} requires 
  always at least 2 ghost zones) order is hardwired in. The two differencings can be selected via
  the parameter {\tt GRHydro::sources\_spatial\_order}.
\item For numerical reasons, namely in order to avoid the presence of time derivatives
  in the source-term computation, the implemented form of the source terms is not
  \eqref{source_terms} directly, but it has been modified as shown in
  the following paragraphs (following a clever idea by Mark Miller (see the {\tt GR3D} code) 
\end{itemize}

%The actual derivation of the source terms is somewhat complex and not
%at all obvious. If anybody comes up with a better way of doing things
%than the below, please alter the documentation. 
In what follows Greek letters range from $0$ to $3$ and roman letters from $1$ to $3$.

For the following computations, we need the expression of some of the 4-Christoffel symbols
$\ {}^{(4)}\Gamma^\rho_{\mu\nu}$ applied to the 3+1 decomposed
variables. In order to remove time derivatives we will frequently make
use of the ADM evolution equation for the 3-metric in the form
%
\begin{equation}
  \label{eq:SourceADMg}
  \partial_t \gamma_{ij} = 2\left(- \alpha K_{ij} + \partial_{(i}
  \beta_{j)} - {}^{(3)}\Gamma^k_{ij} \beta_k \right)\ .
\end{equation}
%
As it is used in what follows, we also recall that $\nabla$ is the
covariant derivative associated with the spatial 3-surface and we note that it is
compatible with the 3-metric:
%
\begin{eqnarray}
\label{compatible_derivative}
\nabla_i\gamma^{jk}=\partial_i\gamma^{jk} + 2{}^{(3)}\Gamma^j_{il}\gamma^{lk} = 0 \ .
\end{eqnarray}
%
We start from the ${}^{(4)}\Gamma^0_{00}$ symbol:
%
\begin{eqnarray}
\label{eq:Gamma000}
{}^{(4)}\Gamma^0_{00} = \frac{1}{2\alpha^2}\Big[
-\partial_t\big(\beta_k\beta^k\big)+2\alpha\partial_t\alpha
+ 2\beta^i\partial_t\beta_i - \beta^i\partial_i\big(\beta_k\beta^k\big) + 2\alpha\beta^i\partial_i\alpha\Big]
\end{eqnarray}
%
and we expand the derivatives as
%
\begin{eqnarray} \label{de_t_beta2}
\partial_t\big(\beta_k\beta^k\big) &=&
\partial_t\big(\gamma_{jk}\beta^j\beta^k\big) = 2\gamma_{jk}\beta^j\partial_t\beta^k +
\beta^j\beta^k\partial_t\gamma_{jk} = \nonumber \\
 &=& 2\beta_k\partial_t\beta^k -2\alpha K_{jk}
\beta^j\beta^k + 2\beta^j\beta^k\partial_j\beta_k - 2{}^{(3)}\Gamma^i_{kj} \beta_i\beta^j\beta^k
\end{eqnarray}
%
and
%
\begin{eqnarray}  \label{de_i_beta2}
\partial_i\big(\beta_k\beta^k\big) =
\partial_i\big(\gamma^{jk}\beta_j\beta_k\big) =
2\gamma^{jk}\beta_j\partial_i\beta_k + \beta_j\beta_k\partial_i\gamma^{jk} =
2\beta_k\partial_i\beta_k -2{}^{(3)}\Gamma^j_{ik}\beta_j\beta^k \ ,
\end{eqnarray}
%
where we have used \eqref{eq:SourceADMg} and
\eqref{compatible_derivative}, respectively.
Inserting \eqref{de_t_beta2} and \eqref{de_i_beta2}, equation
\eqref{eq:Gamma000} becomes
%
\begin{eqnarray}
  \label{eq:Gamma000_final}
{}^{(4)}\Gamma^0_{00} = \frac{1}{\alpha}\Big(\partial_t\alpha +
\beta^i\partial_i\alpha + K_{jk}\beta^j\beta^k \Big)\ .
\end{eqnarray}
%
With the same strategy we then compute
%
\begin{eqnarray}
  \label{eq:SourceChr1a}
  {}^{(4)}\Gamma^0_{i0} & = & - \frac{1}{2\alpha^2} \Big[
      \partial_i (\beta^k \beta_k - \alpha^2) - \beta^j (\partial_i
      \beta_j - \partial_j \beta_i + \partial_t \gamma_{ij}) \Big] = - \frac{1}{\alpha} \Big(\partial_i\alpha - \beta^j K_{ij}\Big)
\end{eqnarray}
%
and
%
\begin{eqnarray}
\label{eq:SourceChr0ij}
  {}^{(4)}\Gamma^0_{ij} & = & - \frac{1}{2\alpha^2} \Big[
      \partial_i\beta_j + \partial_j\beta_i - \partial_t\gamma_{ij} - \beta^k
  (\partial_i\gamma_{kj} + \partial_j\gamma_{ki} - \partial_k\gamma_{ij})\Big] = -
  \frac{1}{\alpha} K_{ij}\ .
\end{eqnarray}
%
Other more straightforward calculations give
%
\begin{alignat}{3}
  \label{eq:SourceS3a}
  {}^{(4)}\Gamma_{00j} &=& {}^{(4)}\Gamma^\nu_{0j}g_{\nu 0} & =  \frac{1}{2} \partial_j \left( \beta_k
    \beta^k - \alpha^2 \right), \\
\nonumber\\
  \label{eq:SourceS3b}
  {}^{(4)}\Gamma_{l0j} &=& {}^{(4)}\Gamma^\nu_{lj}g_{\nu 0} & =  \alpha K_{lj} + \partial_l\beta_j + \partial_j\beta_l - \beta_k{}^{(3)}\Gamma^k_{lj}\ , \\
\nonumber\\
  \label{eq:SourceS3c}
  {}^{(4)}\Gamma_{0lj} &=& {}^{(4)}\Gamma^\nu_{0j}g_{\nu l} & =  -\alpha K_{jl} + \partial_l\beta_j - \beta_k {}^{(3)}\Gamma^k_{lj}\ , \\
\nonumber\\
  \label{eq:SourceS3d}
  {}^{(4)}\Gamma_{lmj} &=& {}^{(4)}\Gamma^\nu_{lj}g_{\nu m} & =  {}^{(3)}\Gamma_{lmj}\ ,
\end{alignat}
%
where~\eqref{eq:SourceADMg} was used to derive \eqref{eq:SourceS3b}
and~\eqref{eq:SourceS3c}.
\subsubsection{Source term for $S_k$}
Now we have all the expressions for calculating the source terms. The
ones for the variables $S_{\,k}$ are
%
\begin{equation}
  \label{eq:SourceS1}
  \big({\mathcal S}_{S_k}\big)_j = T^\mu_\nu \Gamma^\nu_{\mu j} = T^{\mu\nu} \Gamma_{\mu\nu j}\ .
\end{equation}
%
After expanding the derivative in
\eqref{eq:SourceS3a}, the coefficient of the $T^{\ 00}$ term in
\eqref{eq:SourceS1} becomes
%
\begin{eqnarray}
  \label{eq:SourceS4a}
  {}^{(4)}\Gamma_{00j} & = & \frac{1}{2} \beta^l \beta^m \partial_j
  \gamma_{lm} - \alpha \partial_j \alpha + \beta_m \partial_j \beta^m.
\end{eqnarray}
%
The coefficient of the $T^{\,0i}$ term 
% (or the $T^{i0}$ term, as the stress-energy tensor is symmetric) 
is
%
\begin{eqnarray}
  \label{eq:SourceS5a}
  {}^{(4)}\Gamma_{0ij} + {}^{(4)}\Gamma_{i0j} = \partial_j\beta_i = \beta^l \partial_i
  \gamma_{jl} + \gamma_{il} \partial_j \beta^l.
\end{eqnarray}
%
The coefficient of the $T^{\,lm}$ term is simply
%
\begin{eqnarray}
  \label{eq:SourceS6a} 
{}^{(3)}\Gamma_{lmj} =
  \frac{1}{2} \Big (\partial_j\gamma_{ml} + \partial_m\gamma_{jl} - \partial_l\gamma_{mj} \Big).
\end{eqnarray}
%
Finally, summing \eqref{eq:SourceS4a}--\eqref{eq:SourceS6a} we
find
%
\begin{eqnarray}
  \label{eq:SourceS2a} 
\big({\mathcal S}_{S_k}\big)_j & = &
  T^{00}\left( \frac{1}{2} \beta^l \beta^m \partial_j \gamma_{lm} -
  \alpha \partial_j \alpha \right) + T^{0i} \beta^l \partial_j
  \gamma_{il} + T^0_i\partial_j \beta^i + \frac{1}{2} T^{lm}
  \partial_j \gamma_{lm} \ ,
\end{eqnarray}
%
which is the expression implemented in the code. 
%This also has the advantage that all partial derivatives of the shift are collected in
%one term.
% ACTUALLY, NO PARTICULAR EFFORT WAS MADE TO COLLECT THE SHIFT DERIVATIVES...

\subsubsection{Source term for $\tau$}

The source term for $\tau$ is [{\it cf.} \eqref{source_terms}]
%
\begin{equation}
  \label{eq:SourceT1}
  {\mathcal S}_{\tau} = \alpha \left( T^{\mu 0} \partial_{\mu}
  \alpha - \alpha T^{\mu\nu} {}^{(4)}\Gamma^0_{\mu\nu}\right).
\end{equation}
%
For clarity, again we consider separately the terms containing as a
factor the different components of $T^{\mu\nu}$. From
\eqref{eq:Gamma000_final} we find the coefficient of $T^{\,00}$ to be
%
\begin{eqnarray}
  \label{eq:SourceT3a}
\alpha\big(\partial_t \alpha -\alpha {}^{(4)}\Gamma^0_{00}\big) =
-\alpha\big( \beta^i \partial_i \alpha + \beta^k \beta^l K_{kl}\big)\ .
\end{eqnarray}
%
The coefficient of the term $T^{\,0i}$ is given by
% Here we note that we it is T^{0i} + T^{i0}
%
\begin{eqnarray}
  \label{eq:SourceT4a}
  \alpha\big(\partial_i \alpha - 2 \alpha {}^{(4)}\Gamma^0_{i0}\big) =
  2 \alpha\beta^j K_{ij} - \alpha\partial_i \alpha
\end{eqnarray}
%
and, finally, the coefficient for $T^{\,ij}$ is
%
\begin{eqnarray}
  \label{eq:SourceT5a}
  -\alpha^2 {}^{(4)}\Gamma^0_{ij} = \alpha K_{ij}\ .
\end{eqnarray}
The final expression implemented in the code is thus
%
\begin{eqnarray}
  \label{eq:SourceT2a}
  {\mathcal S}_{\tau}  = \alpha\big[ T^{00}\left( \beta^i\beta^j K_{ij} - \beta^i
  \partial_i \alpha \right) +  T^{0i} \left( -\partial_i \alpha + 2
  \beta^j K_{ij} \right) 
+ T^{ij} K_{ij}\big]\ .
\end{eqnarray}

%% original version by Ian. It contains typos, Luca believes
%\subsubsection{Source term for S}
%
%The source terms for $S_j$ are 
%\begin{equation}
%  \label{eq:SourceS1}
%  {\cal S}_{S_j} = \alpha \sqrt{\gamma} T^{\mu\nu}
%  {}^{(4)}\Gamma_{\nu\mu j}.
%\end{equation}
%Ignoring the factor of $\alpha \sqrt{\gamma}$, these source terms are
%coded as
%\begin{eqnarray}
%  \label{eq:SourceS2a}
%  {\cal S}_{S_j} & = & T^{00}\left( \frac{1}{2} \beta^l \beta^m
%    \partial_j \gamma_{lm} - \alpha \partial_j \alpha \right) + \\
%  \label{eq:SourceS2b}
%  &  & T^{0i} \left(  \beta^l \partial_j \gamma_{il} \right) + \\
%  \label{eq:SourceS2c}
%  &  & \frac{1}{2} T^{lm} \partial_j \gamma_{lm} + \\
%  \label{eq:SourceS2d}
%  &  & \frac{\rho h W^2 v_l}{\alpha}\partial_j \beta^l . 
%\end{eqnarray}
%
%In order to get from the first expression to the expression in the
%code we need to calculate the 4-Christoffel symbols ${}^{(4)}\Gamma$
%applied to the 3+1 decomposed variables. In order to remove time
%derivatives we will frequently make use of the ADM evolution equation
%for the 3-metric in the form
%\begin{equation}
%  \label{eq:SourceADMg}
%  \partial_0 \gamma_{ij} = - 2 \left( \alpha K_{ij} + \partial_{(i}
%  \beta_{j)} - {}^{(3)}\Gamma^k_{ij} \beta_k \right),
%\end{equation}
%or equivalent forms.
%
%So, the heart of the calculation is to show that
%\begin{eqnarray}
%  \label{eq:SourceS3a}
%  {}^{(4)}\Gamma_{00j} & = & \frac{1}{2} \partial_j \left( \beta_k
%    \beta^k - \alpha^2 \right), \\
%  \label{eq:SourceS3b}
%  {}^{(4)}\Gamma_{l0j} & = & -\alpha K_{jl} - \beta_{j,l} - \beta_k
%  {}^{(3)}\Gamma^k_{lj}, \\
%  \label{eq:SourceS3c}
%  {}^{(4)}\Gamma_{0mj} & = & \alpha K_{mj} + {}^{(3)}\Gamma^k_{mj}
%  \beta_k, \\
%  \label{eq:SourceS3d}
%  {}^{(4)}\Gamma_{lmj} & = & {}^{(3)}\Gamma_{lmj}.
%\end{eqnarray}
%These are tedious but straightforward, where~(\ref{eq:SourceADMg}) was
%used in equations~(\ref{eq:SourceS3b}) and~(\ref{eq:SourceS3c}).
%
%Substituting these expressions into the original form of
%equation~(\ref{eq:SourceS1}) we find the coefficient of the $T^{00}$
%term to be
%\begin{eqnarray}
%  \label{eq:SourceS4a}
%  {}^{(4)}\Gamma_{00j} & = & \frac{1}{2} \beta^l \beta^m \partial_j
%  \gamma_{lm} - \alpha \partial_j \alpha \\
%  \label{eq:SourceS4b}
%  && + \beta_m \partial_j \beta^m.
%\end{eqnarray}
%Line~(\ref{eq:SourceS4a}) is equivalent to the
%line~(\ref{eq:SourceS2a}) in the code. 
%
%The coefficient of the $T^{0i}+T^{i0}$ term (as the stress-energy
%tensor is symmetric here) is
%\begin{eqnarray}
%  \label{eq:SourceS5a}
%  {}^{(4)}\Gamma_{0ij} + {}^{(4)}\Gamma_{i0j} & = & \beta^l \partial
%  \gamma_{jl} \\
%  \label{eq:SourceS5b}
%  && + \gamma_{il} \partial_j \beta^l.
%\end{eqnarray}
%Similarly to above, line~(\ref{eq:SourceS5a}) is equivalent to the
%line~(\ref{eq:SourceS2b}) in the code. 
%
%The coefficient of the $T^{lm}$ term is given by
%\begin{eqnarray}
%  \label{eq:SourceS6a}
%  {}^{(4)}\Gamma_{lmj} & = & {}^{(3)}\Gamma_{lmj} \\
%  \label{eq:SourceS6b}
%  & = & \frac{1}{2} \gamma_{ml,j} \\ 
%  \label{eq:SourceS6c}
%  && + \frac{1}{2} \left( \gamma_{jl,m} - \gamma_{mj,l} \right).
%\end{eqnarray}
%Again, line~(\ref{eq:SourceS6b}) is equivalent to the
%line~(\ref{eq:SourceS2b}) in the code. 
%
%In each of these sets of coefficients there is an extra line. In the
%case of the coefficient of $T^{lm}$ line~(\ref{eq:SourceS6c}) vanishes
%when contracted with $T^{lm}$ due to the symmetry of the stress-energy
%tensor and the anti-symmetry of line~(\ref{eq:SourceS6c}) with respect
%to $l,m$. However, lines~(\ref{eq:SourceS5b}) and (\ref{eq:SourceS4b})
%do not vanish. Instead, after contraction with the appropriate
%components of the stress-energy tensor, they simplify to form
%line~(\ref{eq:SourceS2d}) in the code. This gathers all partial
%derivatives of the shift in one place.
%
%\subsubsection{Source term for $\tau$}
%
%The derivation of the source term for $\tau$ is a bit more involved.
%
%The source term for $\tau$ is
%\begin{equation}
%  \label{eq:SourceT1}
%  {\cal S}_{\tau} = \alpha \sqrt{\gamma} \left( T^{\mu 0} \partial_{\mu}
%  \alpha - \alpha T^{\mu\nu} {}^{(4)}\Gamma^0_{\mu\nu}\right).
%\end{equation}
%Ignoring the factor of $\alpha \sqrt{\gamma}$, these source terms are
%coded as
%\begin{eqnarray}
%  \label{eq:SourceT2a}
%  {\cal S}_{\tau} & = & T^{00}\left( \beta^i\beta^j K_{ij} - \beta^i
%    \partial_i \alpha \right) + \\
%  \label{eq:SourceT2b}
%  &  & T^{0i} \left( -\partial_i \alpha + 2 \beta^j K_{ij} \right) + \\
%  \label{eq:SourceT2c}
%  &  &  T^{lm} K_{lm}.
%\end{eqnarray}
%
%We consider the coefficient of $T^{00}$ first. In what follows $D$ is
%the covariant derivative associated with the 3-surface. We note that
%it is compatible with the 3-metric, $D_i\gamma_{jk}=0$. Expanding the
%coefficient directly we get
%\begin{eqnarray}
%  \label{eq:SourceT3a}
%  \partial_0 \alpha - \alpha {}^{(4)}\Gamma^0_{00} & = & - \beta^i
%  \partial_i \alpha + \frac{1}{\alpha} \left( \beta^i \partial_0
%    \beta_i - \frac{1}{2} \partial_0 (\beta^k \beta_k) - \frac{1}{2}
%    \beta^i \partial_i (\beta^k \beta_k) \right) \\
%  \label{eq:SourceT3b}
%  & = & -\beta^i \partial_i \alpha + \frac{1}{2\alpha} \left( 2 \beta^i
%    \partial_0 \beta_i - \beta^k \partial_0 \beta_k - \beta_k \partial_0
%    \beta^k - \beta^i D_i (\beta^k \beta_k) \right) \\
%  \label{eq:SourceT3c}
%  & = & -\beta^i \partial_i \alpha + \frac{1}{2\alpha} \left(
%    \gamma_{kl} \beta^l \partial_0 \beta^k - \beta^k \partial_0
%    (\gamma_{kl} \beta^l) - \beta^i \left( \beta_k D_i \beta^k +
%      \beta^k D_i \beta_k \right) \right) \\
%  \label{eq:SourceT3d}
%  & = & -\beta^i \partial_i \alpha + \frac{1}{2\alpha} \left(
%    -\beta^k \beta^l \partial_0 \gamma_{kl} - \beta^i \left( \beta^l
%      D_i \beta_l + \beta^k D_i \beta_k + \beta_k \beta_l D_i
%      \gamma^{kl} \right) \right) \\
%  \label{eq:SourceT3e}
%  & = & -\beta^i \partial_i \alpha + \frac{1}{2\alpha} \left(
%    2 \alpha \beta^k \beta^l K_{kl} - \beta^k \beta^l (D_k \beta_l +
%    D_l \beta_k) + 2 \beta^i \beta^k D_i \beta_k \right) \\
%  \label{eq:SourceT3f}
%  & = & -\beta^i \partial_i \alpha + \beta^k \beta^l K_{kl}.
%\end{eqnarray}
%Again to go from line~(\ref{eq:SourceT3d}) to (\ref{eq:SourceT3e}) we
%used the evolution equation~(\ref{eq:SourceADMg}), expressed in terms
%of the covariant derivative $D$.
%
%To simplify the calculation of the other coefficients we first
%calculate the 4-Christoffel symbol ${}^{(4)}\Gamma^0_{i\nu}$. This is
%given by
%\begin{eqnarray}
%  \label{eq:SourceChr1a}
%  {}^{(4)}\Gamma^0_{i\nu} & = & - \frac{1}{2\alpha^2} \left\{ \left(
%      \partial_i (\beta^k \beta_k - \alpha^2) - \beta^j (\partial_i
%      \beta_j - \partial_j \beta_i + \partial_0 \gamma_{ij}) \right)
%    \delta^0_{\nu} + \right. \\
%    && \left. \left( (\partial_l \beta_i + \partial_i \beta_l -
%      \partial_0 \gamma_{il}) - 2 \beta^j {}^{(3)}\Gamma_{jil} \right)
%    \delta^l_{\nu} \right\} \\
%  \label{eq:SourceChr1b}
%  & = & - \frac{1}{2\alpha^2} \left\{ \left(
%      \partial_i (\beta^k \beta_k - \alpha^2) - \beta^j ( 2 \partial_i
%      \beta_j - 2 \alpha K_{ij} - 2 \beta_k {}^{(3)}\Gamma^k_{ij})
%      \right) \delta^0_{\nu} + \right. \\
%    && \left. \left( (2 \alpha K_{il} + 2 \beta_k
%      {}^{(3)}\Gamma^k_{il}) - 2 \beta^j {}^{(3)}\Gamma_{jil} \right)
%      \delta^l_{\nu} \right\}.
%\end{eqnarray}
%
%The coefficient of $T^{0i} + T^{i0}$ is expanded in a similar
%fashion. Here we note that we only pick up one partial derivative of
%the lapse (as there is a specific ordering on that term) but that
%symmetry gives us two Christoffel symbols,
%\begin{eqnarray}
%  \label{eq:SourceT4a}
%  \partial_i \alpha - 2 \alpha {}^{(4)}\Gamma^0_{i0} & = & \partial_i
%  \alpha + \frac{1}{\alpha} \left( (\beta^k \beta_k - \alpha^2)_{,i} -
%    2 \beta^j (\partial_i \beta_j - \beta_k {}^{(3)}\Gamma^k_{ij} -
%    \alpha K_{ij}) \right) \\
%  \label{eq:SourceT4b}
%  & = & 2 \beta^j K_{ij} - \partial_i \alpha + \frac{1}{\alpha} \left(
%    \partial_i (\beta^k \beta_k) - 2 \beta^j (\partial_i \beta_j -
%    \beta_k {}^{(3)}\Gamma^k_{ij} ) \right). 
%\end{eqnarray}
%We then have to show that the term in brackets vanishes identically.
%Just expanding gives
%\begin{eqnarray}
%  \label{eq:SourceT4c}
%  \left(\dots\right) & = & 2 \beta^k \partial_i \beta_k + \beta^k
%  \beta^l \partial_i \gamma_{kl} - 2 \beta^j \left( \gamma_{jk}
%    \partial_i \beta^k + \beta^k \partial_i \gamma_{jk} - \beta_k
%    \gamma^{kl} (\partial_i \gamma_{lj} + \partial_j \gamma_{il} -
%    \partial_l \gamma_{ij}) \right) \\
%  \label{eq:SourceT4d}
%  & = & \beta^j \beta^l (\partial_j \gamma_{il} - \partial_l
%  \gamma_{ij} )
%\end{eqnarray}
%which is zero by symmetry.
%
%Finally we have the $T^{lm}$ coefficient. Again a simple expansion
%gives
%\begin{eqnarray}
%  \label{eq:SourceT5a}
%  -\alpha {}^{(4)}\Gamma^0_{lm} & = & \frac{1}{2\alpha} \left( 2
%   \alpha K_{lm} + 2 \beta_k {}^{(3)}\Gamma^k_{lm} - 2 \beta^j
%   {}^{(3)} \Gamma_{jlm} \right) \\
%  \label{eq:SourceT5b}
%  & = & K_{lm}.
%\end{eqnarray}
%Once again we had to make use of equation~(\ref{eq:SourceADMg}).


\subsection{Conversion from conservative to primitive variables}
\label{sec:con2prim}

As noted in section~\ref{sec:phys} the variables that are evolved are
the conserved variables $D, S_i, \tau$. But in order to calculate the
fluxes and sources we require the primitive variables $\rho, v_i, P$.
Conversion from primitive to conservative is given analytically by
equation~(\ref{eq:prim2con}). Converting in the other direction is not
possible in a closed form except in certain special circumstances.

There are a number of methods for converting from conservative to
primitive variables; see~\cite{livrevsrrfd}. Here we use a
Newton-Raphson type iteration. If we are using a general equation of
state such as an ideal gas, then we find a root of the pressure
equation. Given an initial guess for the pressure $\bar{P}$ we find
the root of the function
\begin{equation}
  \label{eq:pressure1}
  f = \bar{P} - P(\bar{\rho}, \bar{\epsilon}),
\end{equation}
where the approximate density and specific internal energy are given
by 
\begin{eqnarray}
  \label{eq:press1gives}
  \bar{\rho} & = & \frac{\tilde{D}}{\tilde{\tau} + \bar{P} + \tilde{D}}
  \sqrt{ (\tilde{\tau} + \bar{P} + \tilde{D})^2 - S^2 }, \\
  \bar{W} & = & \frac{\tilde{\tau} + \bar{P} + \tilde{D}}{\sqrt{
      (\tilde{\tau} + \bar{P} + \tilde{D})^2 - S^2 }}, \\
  \bar{\epsilon} & = & \tilde{D}^{-1} \left( \sqrt{ (\tilde{\tau} +
      \bar{P} + \tilde{D})^2 - S^2 } - \bar{P} \bar{W} - \tilde{D}
  \right). 
\end{eqnarray}
Here the conserved variables are all ``undensitized'', {\it e.g.},
\begin{equation}
  \label{eq:undens}
  \tilde{D} = \gamma^{-1/2} D,
\end{equation}
where $\gamma$ is the determinant of the 3-metric, and $S^2$ is given
by 
\begin{equation}
  \label{eq:s2}
  S^2 = \gamma^{ij}\tilde{S}_i\tilde{S}_j.
\end{equation}

In order to perform a Newton-Raphson iteration we need the derivative
of the function with respect to the dependent variable, here the
pressure. This is given by
\begin{equation}
  \label{eq:df}
  f' = 1 - \frac{\partial P}{\partial \rho}\frac{\partial
  \rho}{\partial P} - \frac{\partial P}{\partial
  \epsilon}\frac{\partial \epsilon}{\partial P}, 
\end{equation}
where $\frac{\partial P}{\partial \rho}$ and $\frac{\partial
  P}{\partial \epsilon}$ given by calls to {\tt EOS\_Base}, and
\begin{eqnarray}
  \label{eq:df2}
  \frac{\partial \rho}{\partial P} & = & \frac{\tilde{D}
      S^2}{\sqrt{(\tilde{\tau} + \bar{P} + \tilde{D})^2 -
      S^2}(\tilde{\tau} + \bar{P} + \tilde{D})^2}, \\
  \frac{\partial \epsilon}{\partial P} & = & \frac{\bar{P}
      S^2}{\rho\left((\tilde{\tau} + \bar{P} + \tilde{D})^2 -
      S^2\right)(\tilde{\tau} + \bar{P} + \tilde{D})}. \\
\end{eqnarray}

For a polytropic type equation of state, the function is given by
\begin{equation}
  \label{eq:polyf}
  f = \bar{\rho}\bar{W} - \tilde{D},
\end{equation}
where $\bar{\rho}$ is the variable solved for, the pressure, specific
internal energy and enthalpy $\bar{h}$ are set from the EOS and the
Lorentz factor is found from
\begin{equation}
  \label{eq:polyw}
  \bar{W} = \sqrt{1 + \frac{S^2}{(\tilde{D}\bar{h})^2}}.
\end{equation}
The derivative is given by
\begin{equation}
  \label{eq:dpolyf}
  f' = \bar{W} - \frac{\bar{\rho}S^2 \bar{h}'}{\bar{W} \tilde{D}^2
  \bar{h}^3}, 
\end{equation}
where
\begin{equation}
  \label{eq:dpolyenth}
  \bar{h}' = \bar{\rho}^{-1}\frac{\partial P}{\partial \rho}.
\end{equation}

\subsection{A note on the Roe and Marquina Riemann Solvers}
\label{sec:rsnote}

Finding the Roe or Marquina fluxes as given is
section~\ref{sec:riemann} requires the left eigenvectors to either be
supplied analytically or calculated numerically. 

When this is done by inverting the matrix of right
eigenvectors, in the actual code this is combined with the calculation
of, {\it e.g.}, the characteristic jumps $\Delta {\bf w}$.
Normally the eigenvalues and vectors are ordered lexicographically.
However for the polytropic equation of state one of the equations is
redundant, so the matrix formed by these eigenvectors is linearly
dependent and hence singular. It turns out that this is only a minor
problem; by rearranging the order of the eigenvalues and vectors it is
possible to numerically invert the matrix. 
This means that no specific ordering of the eigenvalues should be
assumed. It also explains the slightly strange ordering in the
routines {\tt GRHydro\_EigenProblem*.F90}.

The current default is that the left eigenvectors are calculated
analytically - for the expressions see Font~\cite{livrevgrrfd}. For
both the Roe and the Marquina solvers an optimized version of the flux
calculation has been implemented. For more details on the analytical
form and the optimized flux calculation see Ib{\'a}{\~n}ez et
al.~\cite{Iban01}, Aloy et al.~\cite{Aloy99} and Frieben et
al.~\cite{Frie02}.

\subsection{The atmosphere}
\label{sec:atmosphere}

In simulations of compact objects, often the matter is located only on a (small) portion of the
numerical grid. In fact, over much of the evolved domain the physical situation is likely to be
sufficiently well approximated by vacuum. However, in the vacuum limit the continuity equations
describing the fluid break down. The speed of sound tends to the speed of light and everything fails
(especially the conversion from conserved to primitive variables).

To avoid this problem it is customary to introduce an atmosphere. In our implementation, this is a
low-density region surrounding the compact objects and initially it has no velocity and is in equilibrium. The
introduction of an atmosphere is managed by the initial data thorns.

However {\tt GRHydro} itself also knows about the atmosphere, of course. If the conserved variables
$D$ or $\tau$ are beneath some minimum value, or an evolution step might push them below such a
value, then the relevant cell is not evolved. Also, if the density should fall below a minimum value
in the routine that converts from conservative to primitive variables, all the variables are reset
to the values adopted for the atmosphere.

Probably the hardest part of using {\tt GRHydro} is to correctly set these
atmosphere values. In the current implementation the atmosphere is
used in three separate places. These are
\begin{enumerate}
\item {\bf Set up of the initial data}. Initial-data routines should set an atmosphere consistent
  with the one that will be evolved.
\item {\bf In the routine that converts from conserved variables to
    primitive variables}. This is where the majority of the atmosphere
  resets will occur. 

  If the equation of state is polytropic then an
  attempt is made to convert to primitive variables. If the iterative
  algorithm returns a negative (and hence unphysical) value of $\rho$,
  then $\rho$ is reset to the atmosphere value, the velocities are set
  to zero, and $P$, $\epsilon$, $S_i$ and $\tau$ are reset to be
  consistent with $\rho$ (and $D$). Note that even though the
  polytropic equation of state gives us sufficient information to
  calculate a consistent value of $D$, this is not done.

  If the equation of state is the more general type (such as that of an ideal fluid) and if $\rho$
  is less than the specified minimum, then, as above, we assume we are in the atmosphere and call
  the routine that changes from the conserved to the primitive variables for the polytrope.

\item {\bf When applying the update}. If the calculated update terms
  for a specific cell would lead to either $D$ or $\tau$ becoming
  negative, then two steps are taken. First, we do not update this
  specific cell. Second, the data in this cell is reset to be the
  atmosphere. 
\end{enumerate}

The reason why the routine that converts to the primitive variables
does not ensure that $D$ is consistent with the other variables is
practical rather than accurate. If the value of the variables is set
such that they all lie precisely on the atmosphere, then small errors
(typically initially of the order of $10^{-25}$ for a $64^3$-point TOV star
in octant symmetry) would move certain cells above the atmosphere
values. Combined with the necessary atmosphere treatment this leads to
high-frequency noise. This will lead to waves of matter falling onto
the star. Despite their extremely low density (typically only an
order of magnitude higher than the floor) they will result in visible
secondary overtones in the oscillations of, {\it e.g.}, the central
density. 

The parameters controlling the atmosphere are the following.
\begin{itemize}
\item {\tt GRHydro::rho\_abs\_min}. An absolute value for $\rho$ in the
  atmosphere. Defaults to -1. Any negative value will be ignored, and
  the value of {\tt rho\_rel\_min} used instead.
\item {\tt GRHydro::rho\_rel\_min}. A relative value for $\rho$ in the
  atmosphere. Defaults to $10^{-7}$. Only used if {\tt rho\_abs\_min}
  is negative, which is the default behaviour. The actual value of the
  atmosphere will be $\rho =${\tt rho\_rel\_min}$\times${\tt
    GRHydro\_rho\_max}, where {\tt GRHydro::GRHydro\_rho\_max}
  is a variable containing the maximum value of $\rho$ on the numerical grid at time zero.
\item {\tt initial\_rho\_abs\_min}. An absolute value for rho in the initial atmosphere. It is used
  only by initial data routines. Unused if negative.
\item {\tt initial\_rho\_rel\_min}. A relative (to the initial maximum rest-mass density) value for rho
  in the atmosphere. It is used only by initial data routines and only if it is positive and {\tt
  initial\_rho\_abs\_min} is negative.
\item {\tt initial\_atmosphere\_factor}. A relative (to the initial atmosphere) value for rho in the
  atmosphere. It is used only by initial data routines. It multiplies the atmosphere value used by
  the initial data solver. Unused if negative.
\item {\tt GRHydro\_atmo\_tolerance}. A parameter useful mostly in mesh-refined simulations. A point
  is set to the atmosphere values in the conservative to primitive routines if its rest-mass density
  is such that $\rho <$ {\tt GRHydro\_rho\_min}$*(1+${\tt GRHydro\_atmo\_tolerance}). This avoids
  occasional spurious oscillations in ({\tt Carpet}) buffer zones lying in the atmosphere (because
  prolongation happens on conserved variables).
\end{itemize}

The motivation for these parameters referring only to the initial data is that it is sometimes best
to set the initial atmosphere to slightly below the atmosphere cutoff used during evolution, as this
avoids truncation error and metric evolution leading to low density waves travelling across the
atmosphere.

The routines essential to the atmosphere are contained in {\tt
GRHydro\_Minima.F90, GRHydro\_Con2Prim.F90, GRHydro\_UpdateMask.F90}. 

\subsection{Advection of passive scalars ('tracers')}
\label{sec:tracer}

For some astrophysical problems it is necessary to advect passive
compositional scalars such as the electron fraction $Y_e$ (number of
electrons per baryon). For a generic tracer $X_k$, the evolution
equation looks like

%%%\begin{eqnarray}
%%%  \label{eq:tracer}
%%%   D &=& \sqrt{\gamma}W\rho \nonumber \\
%%%   S^i &=& \sqrt{\gamma} \rho h W^2 v^i \nonumber \\
%%%   \tau &=& \sqrt{\gamma}\left( \rho h W^2 - p\right) - D, 
%%%\end{eqnarray}

\begin{equation}
  \label{eq:tracer}
  \partial_t { ( D X_k )} + \partial_{x^j} {\bf f}^{(j)} ({D X_k}) = 0\, ,
\end{equation}
where $D$ is the generalized particle number density as defined in
Eq.~(\ref{eq:prim2con}). {\tt GRHydro} currently supports any number of
independent tracer variables. The following parameters have to be
set to use the tracers:

\begin{itemize}
  \item {\tt GRHydro::evolve\_tracer}. Boolean. Set to {\tt yes} if
    you want the tracers to be active.
  \item {\tt GRHydro::number\_of\_tracers}. Integer. Defaults to 0. To
    use tracers, set to at least 1.
\end{itemize}

Note, that your initial data thorn must set initial data for
{\tt GRHydro::tracer[k]} and {\tt GRHydro::cons\_tracer[k]} for all tracers
you want to advect. {\tt GRHydro::cons\_tracer[k]} stores $D X_k$.

\subsubsection{Implementation and Limitations}

\begin{itemize}
  \item Reconstruction: Currently only TVD and PPM reconstruction of the
    tracers $X_k$ are implemented. Since for most astrophysical problems one
    will associate the tracer with some compositional variable it might
    be better to reconstruct $\rho X_k$.

  \item Riemann Solvers: Only HLLE and Marquina are supported. In HLLE,
    the fluxes as given in Eq.~(\ref{eq:tracer}) are computed for each tracer.
    In the Marquina solver, we multiply the $D$-flux by each tracer to get
    the individual tracer fluxes according to the following prescription:

    \begin{equation}
      \label{eq:marquinatracer}
      \begin{array}[l]{l}
	{\bf If}\ \, F_{i+1/2} (D) > 0 \,\, {\bf then} \\
	  \qquad \qquad \begin{array}[c]{l}
	    \qquad \qquad F_{i+1/2} (D X_k) = X_{k_i}^+ F_{i+1/2} (D)
	  \end{array} \\
		 {\bf else} \\
		 \qquad \qquad \begin{array}[c]{r c l}
		   \qquad \qquad F_{i+1/2} (D X_k) = X_{k_{i+1}}^- F_{i+1/2} (D)
		 \end{array} \\
			{\bf end if} \\
      \end{array} \\
    \end{equation}

    The above was suggested by Miguel Aloy and first implemented and
    tested by Harry Dimmelmeier in his code (CoCoNuT), then by Christian~D.~Ott
    in {\tt GRHydro}.

\end{itemize}


\section{History}

The approximate time line is something like this:
\begin{itemize}
\item ~1995: Valencia group hydrodynamics code, fixed spacetime.
\item ~1997: Ported to Cactus, extensively rewritten for the Binary
  Neutron Star Grand Challenge. Primarily written by Mark Miller.
  Released as {\tt GR3D} as public domain code.
\item ~1998-: Developed as Cactus thorn {\tt MAHC} inside the
  GR{\tt Astro\_Hydro} arrangement at Washington University, primarily by
  Mark Miller.
\item 2002-2008: {\tt Whisky} written based on {\tt GR3D}.
\item 2008-: {\tt GRHydro} based on the public version of {\tt Whisky}
\end{itemize}

This is necessarily only a sketch; many people have contributed to
the history of this code, and the present authors were not around for most of it...

\subsection{Thorn Source Code}

This was initially written by Luca Baiotti, Ian Hawke and Pedro
Montero with considerable assistance from Luciano Rezzolla, Toni Font,
Nick Stergioulas and Ed Seidel. This led to the basic {\tt GRHydro} thorns,
{\tt GRHydro} itself, {\tt GRHydro\_Init\_Data} and {\tt GRHydro\_RNSID}.

Since then most of the maintenance has been done by Ian Hawke, Luca Baiotti and Frank L\"offler. Various
people have contributed to the development. In particular
\begin{itemize}
\item The PPM reconstruction methods were written by Ian Hawke heavily
  based on the code of Toni Font. They were later expanded by Christian D. Ott and Luca Baiotti.
\item The Roe and Marquina solvers were made considerably more
  efficient thanks to Joachim Frieben.
\item The current atmosphere algorithm is a mixture of ideas from the
  {\tt GR3D} code, Luciano Rezzolla, Toni Font and Nick
  Stergioulas. The current setup was written by Ian Hawke and Luca Baiotti.
\item The 1-dimensional TOV solver {\tt GRHydro\_TOVSolver} was written
  by Ian Hawke based on a short paper by Thomas Baumgarte.
%\item The use of the equation of state, in particular the routines to
%  ensure the initial hydrodynamic consistency, were due to the work of
%  Harald Dimmelmeier and Christian Ott.
\end{itemize}

\subsection{Thorn Documentation}

This documentation was first written largely by Ian Hawke and Scott Hawley in 2002. Long due,
rather necessary and considerably large updates were made in 2008 by Luca Baiotti.


\subsection{Acknowledgements}

As already mentioned, the history behind this code leads to a long list
of people to be acknowledged.

Firstly, without the work of the Valencia group this sort of code
would be impossible. 

Secondly, the incomparable work of Mark Miller and the Washington
University - AEI Collaboration in producing the {\tt GR3D} and {\tt
GRAstro\_Hydro} codes gave an essential benchmark to aim for, and
encouragement that it was possible!

Thirdly, the support of the Cactus team, especially Tom Goodale,
Gabrielle Allen and Thomas Radke made life a
lot easier.

Finally, for their work in coding, ideas and suggestions, or just
plain encouragement, we would like to thank all at the AEI and in the
EU Network, especially Toni Font, Luciano Rezzolla, Nick Stergioulas,
Ed Seidel, Carsten Gundlach and Jos\'e-Maria Ib{\'a}{\~n}ez.

Originally Ed Seidel and then Luciano Rezzolla and Gabrielle Allen and many others 
have been granting (in addition to valuable
scientific advice) financial support and human resources to the development of the code.

\begin{thebibliography}{20}

\bibitem{Aloy99b}
Aloy M.A., Ib{\'a}nez J.M., Mart\'\i J.M., M{\"u}ller E.
\newblock {\em Astroph. J. Supp.\/}, {\bf 122}: 151 (1999).

\bibitem{Aloy99} 
M.~A. Aloy, J.~A. Pons, and J.~M. Ib{\'a}{\~n}ez.
\newblock An efficient implementation of flux formulae in multidimensional
  relativistic hydrodynamical codes.
\newblock {\em Comput. Phys. Commun.}, {\bf 120}:\penalty0 115--121, 1999.

\bibitem{Banyuls97}
Banyuls F., Font J.A., Ib{\'a}nez J.M., Mart\'{\i} J.M., Miralles J.A.
\newblock {\em Astrophys. J.\/}, {\bf 476}: 221 (1997).

\bibitem{ppm}
P. Colella and P.~R. Woodward.
\newblock The {P}iecewise {P}arabolic {M}ethod ({PPM}) for
{G}as-{D}ynamical {S}imulations.
\newblock {\em J. Comput. Phys.}, {\bf 54}, 174--201, 1984.

\bibitem{Cook00}
G. Cook
\newblock Initial Data for Numerical Relativity
\newblock {\em Living Rev. Relativity}, {\bf 3}, 2000.
\newblock [Article in on-line journal], cited on 31/08/02,
  http://www.livingreviews.org/ Articles/Volume3/2000-5cook/index.html.

\bibitem{Marquina1}
R.~Donat and A.~Marquina.
\newblock Capturing shock reflections: An improved flux formula.
\newblock {\em J. Comput. Phys.}, {\bf 125}:\penalty0 42--58, 1996.

\bibitem{Einfeldt88}
Einfeldt B.
\newblock {\em Journal of Computational Physics\/}, {\bf 25}: 294 (1988).

\bibitem{Eulderink94}
Eulderink F., Mellema G.
\newblock {\em Astron. Astrophys.\/}, {\bf 284}: 652 (1994).

\bibitem{livrevgrrfd}
J.~A. Font.
\newblock Numerical hydrodynamics in {G}eneral {R}elativity.
\newblock {\em Living Rev. Relativity}, {\bf 3}, 2000.
\newblock [Article in on-line journal], cited on 31/07/01,
  http://www.livingreviews.org/ Articles/Volume3/2000-2font/index.html.

\bibitem{Frie02}
J. Frieben, J.~M. Ib{\'a}{\~n}ez, and J. Pons.
\newblock {\em in preparation}

\bibitem{eno}
A.~Harten, B.~Engquist, S.~Osher, and S.~R. Chakravarthy.
\newblock Uniformly high order accurate essentially non-oscillatory schemes,
  {III}.
\newblock {\em J. Comput. Phys.}, {\bf 71}:\penalty0 231--303, 1987.

\bibitem{Iban01}
J.~M. Ib{\'a}{\~n}ez et al.
\newblock in {\em Godunov Methods: Theory and Applications}.
\newblock New York, 485--503, (2001)

\bibitem{Tadmor00}
A. Kurganov and E. Tadmor.
\newblock {\em New high-resolution central schemes for nonlinear conservation laws and
  convection-diffusion equations}.
\newblock {\em J. Comput. Phys.}, {\bf 160}:241, 2000.

\bibitem{laney}
C.~B. Laney.
\newblock {\em Computational {G}asdynamics}.
\newblock Cambridge University Press, 1998.

\bibitem{leveque}
R.~J. LeVeque.
\newblock {N}onlinear conservation laws and finite volume methods for
  astrophysical fluid flow.
\newblock In O.~Steiner and A.~Gautschy, editors, {\em {C}omputational
  {M}ethods for {A}strophysical {F}luid {F}low}. Springer-Verlag, 1998.

\bibitem{livrevsrrfd}
J.~M. Mart{\'{\i}} and E.~M{\"u}ller.
\newblock Numerical hydrodynamics in {S}pecial {R}elativity.
\newblock {\em Living Rev. Relativity}, {\bf 2}, 1999.
\newblock [Article in on-line journal], cited on 31/7/01,
  http://www.livingreviews.org/Articles/Volume2/1999-3marti/index.html.

\bibitem{Quirk}
J.~J. Quirk.
\newblock A contribution to the great {R}iemann solver debate.
\newblock {\em Int. J. Numer. Methods Fluids}, {\bf 18}:\penalty0 555--574,
  1994.

\bibitem{Roe81}
Roe P.L.
\newblock {\em J. Comput. Phy.\/}, {\bf 43}: 357 (1981).

\bibitem{shueno}
C. Shu.
\newblock {H}igh {O}rder {ENO} and {WENO} {S}chemes for
{C}omputational {F}luid {D}ynamics.
\newblock In T.~J. Barth and H. Deconinck, editors {\em High-Order
  Methods for Computational Physics}. Springer, 1999.
\newblock A related on-line version can be found under {\em Essentially
  {N}on-{O}scillatory and {W}eighted {E}ssentially {N}on-{O}scillatory
  {S}chemes for {H}yperbolic {C}onservation {L}aws} at {\tt
  http://www.icase.edu/library/reports/rdp/97/97-65RDP.tex.refer.html}. 

\bibitem{toro}
E.~F. Toro.
\newblock {\em {R}iemann {S}olvers and {N}umerical {M}ethods for {F}luid
  {D}ynamics}.
\newblock {S}pringer-{V}erlag, 2nd edition, 1999.

\bibitem{Moesta:2013dna}
  P.~Mösta {\it et al.},
  `GRHydro: A new open source general-relativistic magnetohydrodynamics code for the Einstein Toolkit,''
  Class.\ Quant.\ Grav.\  {\bf 31}, 015005 (2014)
  doi:10.1088/0264-9381/31/1/015005
  [arXiv:1304.5544 [gr-qc]].

\end{thebibliography}

% Do not delete next line
% END CACTUS THORNGUIDE

\end{document}
